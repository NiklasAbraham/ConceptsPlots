% Act 0 — Software Literacy Primer
\section{Orientation: Software Foundations Executives Need}

% ==============================================================================
% 0.1 The "Stack Map" (1 slide)
% ==============================================================================

\begin{frame}{The Technology Stack: AI in Context}
    \framesubtitle{Understanding where AI fits in your organization}
    
    {\color{TuebingenGray} AI is not magic—it's software built on data, running on infrastructure, serving business processes.}
    
    \vspace{0.8em}
    \begin{center}
        \begin{tikzpicture}[
            box/.style={draw=TuebingenAnthrazit, rounded corners=3pt, minimum width=2.8cm, minimum height=0.9cm, align=center, font=\small},
            arrow/.style={->, thick, TuebingenGold}
        ]
            % Stack layers (bottom to top)
            \node[box, fill=TuebingenBeige] (infra) at (0,0) {Infrastructure\\{\tiny CPU/GPU, Cloud}};
            \node[box, fill=TuebingenBeige] (data) at (0,1.3) {Data Systems\\{\tiny Databases, Pipelines}};
            \node[box, fill=TuebingenBeige] (software) at (0,2.6) {Software Systems\\{\tiny APIs, Services}};
            \node[box, fill=TuebingenRot!20] (ai) at (0,3.9) {ML/AI Models\\{\tiny Training, Inference}};
            \node[box, fill=TuebingenBeige] (product) at (0,5.2) {Product Interface\\{\tiny Apps, Dashboards}};
            \node[box, fill=TuebingenBeige] (govern) at (0,6.5) {Monitoring \& Governance\\{\tiny Compliance, Ops}};
            
            % Arrows
            \draw[arrow] (infra) -- (data);
            \draw[arrow] (data) -- (software);
            \draw[arrow] (software) -- (ai);
            \draw[arrow] (ai) -- (product);
            \draw[arrow] (product) -- (govern);
            
            % Business context
            \node[anchor=west, align=left, font=\small, TuebingenAnthrazit] at (3,3.9) {
                \textbf{Executive Reality:}\\[0.3em]
                $\blacktriangleright$ AI requires \textit{all} layers working\\[0.2em]
                $\blacktriangleright$ Model is often < 20\% of effort\\[0.2em]
                $\blacktriangleright$ Data quality gates success\\[0.2em]
                $\blacktriangleright$ Governance is not optional
            };
        \end{tikzpicture}
    \end{center}
    
    \vspace{0.5em}
    \begin{center}
        \small\color{TuebingenRot} \textbf{Key insight:} "AI" is software + data + evaluation. Invest across the stack.
    \end{center}
\end{frame}

% ==============================================================================
% 0.2 Programming Languages: What Exists and Why It Matters (2 slides)
% ==============================================================================

\begin{frame}{Programming Languages: The Landscape}
    \framesubtitle{Why language choice matters for your AI initiatives}
    
    {\color{TuebingenGray} Different languages serve different purposes. Understanding this helps evaluate team composition and vendor choices.}
    
    \vspace{0.5em}
    \begin{columns}[T]
        \column{0.48\textwidth}
        \textbf{\color{TuebingenRot}Data \& ML Ecosystem:}
        \begin{itemize}
            \setlength\itemsep{0.3em}
            \item \textbf{Python} — dominant for ML/AI
                \begin{itemize}
                    \item Rich libraries (TensorFlow, PyTorch)
                    \item Rapid prototyping
                    \item Data science standard
                \end{itemize}
            \item \textbf{R / MATLAB} — statistical analysis niches
        \end{itemize}
        
        \vspace{0.5em}
        \textbf{\color{TuebingenGold}Enterprise \& Backend:}
        \begin{itemize}
            \setlength\itemsep{0.3em}
            \item \textbf{Java} — enterprise systems, stability
            \item \textbf{Go} — cloud infrastructure, concurrency
            \item \textbf{C\#} — Microsoft ecosystem
        \end{itemize}
        
        \column{0.48\textwidth}
        \textbf{\color{TuebingenGreen}Performance-Critical:}
        \begin{itemize}
            \setlength\itemsep{0.3em}
            \item \textbf{C/C++} — model runtimes, systems
            \item \textbf{Rust} — safety + performance
            \item \textbf{CUDA} — GPU programming
        \end{itemize}
        
        \vspace{0.5em}
        \textbf{\color{TuebingenCyan}Product Interfaces:}
        \begin{itemize}
            \setlength\itemsep{0.3em}
            \item \textbf{JavaScript/TypeScript} — web, full-stack
            \item \textbf{Swift/Kotlin} — mobile apps
        \end{itemize}
        
        \vspace{0.5em}
        \textbf{\color{TuebingenAnthrazit}Specialized:}
        \begin{itemize}
            \setlength\itemsep{0.3em}
            \item \textbf{Haskell/Scala} — type safety, correctness
            \item \textbf{SQL} — data querying (ubiquitous)
        \end{itemize}
    \end{columns}
\end{frame}

\begin{frame}{Why Language Choice Matters for AI}
    \framesubtitle{Ecosystems, talent, and AI assistance quality}
    
    \vspace{0.5em}
    \begin{columns}[T]
        \column{0.55\textwidth}
        \textbf{The "Gravity Well" Effect:}
        \begin{itemize}
            \setlength\itemsep{0.5em}
            \item ML research concentrates in \textbf{Python}
            \item Enterprise gravity in \textbf{Java/Go}
            \item Performance work in \textbf{C++/Rust}
            \item Each ecosystem has its own:
                \begin{itemize}
                    \item Package libraries
                    \item Community expertise
                    \item Hiring pool
                \end{itemize}
        \end{itemize}
        
        \column{0.42\textwidth}
        \begin{theorembox}{AI Coding Assistant Quality}
            LLM coding tools perform best where \textbf{training data is abundant}.
            
            \vspace{0.3em}
            {\small
            \textbf{Strong support:} Python, JavaScript, Java, Go
            
            \textbf{Moderate:} C++, Rust, TypeScript
            
            \textbf{Weaker:} MATLAB, R, niche languages}
        \end{theorembox}
    \end{columns}
    
    \vspace{0.8em}
    \begin{center}
        \small\color{TuebingenRot} \textbf{Executive takeaway:} Language choice shapes experimentation speed, maintainability, hiring, and AI-assistance leverage.
    \end{center}
\end{frame}

% ==============================================================================
% 0.3 One Identical Example in Three Languages (3 slides)
% ==============================================================================

\begin{frame}[fragile]{Code Comparison: The Same Task in Three Languages}
    \framesubtitle{Task: Load CSV, compute summary, detect anomalies, output JSON}
    
    {\color{TuebingenGray} Seeing the same logic expressed differently reveals language philosophies.}
    
    \vspace{0.5em}
    \begin{columns}[T]
        \column{0.48\textwidth}
        \begin{codebox}
            {\small\textbf{\color{TuebingenGreen}Python} — Concise, library-rich}
            \vspace{0.3em}
\begin{lstlisting}[style=pythonstyle, basicstyle=\ttfamily\tiny]
import pandas as pd
import json

# Load and analyze
df = pd.read_csv("transactions.csv")
summary = {
    "total": df["amount"].sum(),
    "mean": df["amount"].mean(),
    "count": len(df)
}

# Detect anomalies (simple rule)
threshold = summary["mean"] * 3
anomalies = df[df["amount"] > threshold]
summary["anomalies"] = len(anomalies)

# Output
with open("report.json", "w") as f:
    json.dump(summary, f)
\end{lstlisting}
        \end{codebox}
        
        \column{0.48\textwidth}
        {\small\color{TuebingenAnthrazit}
        \textbf{Characteristics:}
        \begin{itemize}
            \setlength\itemsep{0.2em}
            \item 15 lines of code
            \item Rich standard library
            \item Readable, minimal boilerplate
            \item Dynamic typing (flexible)
            \item Dominant in data science
        \end{itemize}
        
        \vspace{0.5em}
        \textbf{Trade-offs:}
        \begin{itemize}
            \setlength\itemsep{0.2em}
            \item Slower runtime than compiled
            \item Type errors found at runtime
            \item GIL limits parallelism
        \end{itemize}
        }
    \end{columns}
\end{frame}

\begin{frame}[fragile]{Code Comparison: Java — Enterprise Standard}
    \framesubtitle{Same task: More structure, explicit types, verbose}
    
    \begin{columns}[T]
        \column{0.55\textwidth}
        \begin{codebox}
            {\small\textbf{\color{TuebingenGreen}Java} — Explicit, structured}
            \vspace{0.2em}
\begin{lstlisting}[style=javastyle, basicstyle=\ttfamily\tiny]
public class TransactionAnalyzer {
    public static void main(String[] args) {
        List<Transaction> txns = loadCSV("transactions.csv");
        
        double total = txns.stream()
            .mapToDouble(Transaction::getAmount)
            .sum();
        double mean = total / txns.size();
        double threshold = mean * 3;
        
        long anomalyCount = txns.stream()
            .filter(t -> t.getAmount() > threshold)
            .count();
        
        Summary summary = new Summary(
            total, mean, txns.size(), anomalyCount);
        
        ObjectMapper mapper = new ObjectMapper();
        mapper.writeValue(
            new File("report.json"), summary);
    }
}
\end{lstlisting}
        \end{codebox}
        
        \column{0.42\textwidth}
        {\small\color{TuebingenAnthrazit}
        \textbf{Characteristics:}
        \begin{itemize}
            \setlength\itemsep{0.2em}
            \item ~25 lines (plus class definitions)
            \item Static typing (compile-time safety)
            \item Explicit structure
            \item Enterprise conventions
            \item Long-lived, maintainable codebases
        \end{itemize}
        
        \vspace{0.5em}
        \textbf{Trade-offs:}
        \begin{itemize}
            \setlength\itemsep{0.2em}
            \item More boilerplate
            \item Slower iteration
            \item Steeper learning curve
        \end{itemize}
        }
    \end{columns}
\end{frame}

\begin{frame}[fragile]{Code Comparison: Go — Modern Systems Language}
    \framesubtitle{Same task: Explicit error handling, built for services}
    
    \begin{columns}[T]
        \column{0.55\textwidth}
        \begin{codebox}
            {\small\textbf{\color{TuebingenGreen}Go} — Explicit, concurrent-ready}
            \vspace{0.2em}
\begin{lstlisting}[style=gostyle, basicstyle=\ttfamily\tiny]
func analyzeTransactions() error {
    file, err := os.Open("transactions.csv")
    if err != nil {
        return fmt.Errorf("open: %w", err)
    }
    defer file.Close()
    
    txns, err := parseCSV(file)
    if err != nil {
        return fmt.Errorf("parse: %w", err)
    }
    
    var total float64
    for _, t := range txns {
        total += t.Amount
    }
    mean := total / float64(len(txns))
    threshold := mean * 3
    
    var anomalies int
    for _, t := range txns {
        if t.Amount > threshold {
            anomalies++
        }
    }
    
    summary := Summary{Total: total, Mean: mean,
        Count: len(txns), Anomalies: anomalies}
    return writeJSON("report.json", summary)
}
\end{lstlisting}
        \end{codebox}
        
        \column{0.42\textwidth}
        {\small\color{TuebingenAnthrazit}
        \textbf{Characteristics:}
        \begin{itemize}
            \setlength\itemsep{0.2em}
            \item ~30 lines
            \item \textbf{Explicit error handling}
            \item Compiled, fast execution
            \item Built-in concurrency
            \item Cloud/DevOps standard
        \end{itemize}
        
        \vspace{0.5em}
        \textbf{Go Philosophy:}
        \begin{itemize}
            \setlength\itemsep{0.2em}
            \item "Errors are values"
            \item Simplicity over cleverness
            \item Designed for services
        \end{itemize}
        
        \vspace{0.5em}
        \color{TuebingenRot}\small\textit{Used by: Docker, Kubernetes, most cloud infrastructure}
        }
    \end{columns}
\end{frame}

% ==============================================================================
% 0.5 Editors and AI Coding Copilots (1 slide)

% ------------------------------------------------------------------------------
\begin{frame}{Development Tools: Editors and AI Coding Assistants}
    \framesubtitle{The delivery vehicle for AI in engineering}
    {\color{TuebingenGray} Modern editors are where AI meets developers—and where governance matters most.}
    \vspace{0.5em}
    \begin{columns}[T]
        \column{0.48\textwidth}
        	extbf{Popular Development Environments:}
        \begin{itemize}
            \setlength\itemsep{0.4em}
            \item \textbf{VS Code}: Free, open source, huge extension ecosystem, strong AI integration (Copilot, etc.), cross-platform. Used by individuals and enterprises.
            \item \textbf{Cursor}: AI-native fork of VS Code, built-in copilots, context window navigation, paid plans for advanced AI features.
            \item \textbf{JetBrains Suite} (IntelliJ, PyCharm, etc.): Paid, advanced refactoring, deep language support, enterprise features, strong static analysis, AI assistant (paid add-on).
            \item \textbf{Vim/Neovim}: Free, highly customizable, keyboard-driven, used on servers and by power users. AI plugins available.
        \end{itemize}
        \vspace{0.5em}
        	extbf{AI Coding Assistant Capabilities:}
        \begin{itemize}
            \setlength\itemsep{0.3em}
            \item Code completion (line/block)
            \item Refactoring suggestions
            \item Test generation
            \item Documentation writing
            \item Code search and explanation
            \item Agentic workflows (bounded)
        \end{itemize}
        \column{0.48\textwidth}
        \begin{theorembox}{Governance Implications}
            AI coding tools require explicit policies:
            \vspace{0.3em}
            \begin{itemize}
                \setlength\itemsep{0.3em}
                \item \textbf{Secrets} — API keys, credentials exposure
                \item \textbf{IP/Licensing} — training data, code ownership
                \item \textbf{Security} — vulnerable code suggestions
                \item \textbf{Auditability} — who wrote what?
                \item \textbf{Data residency} — where does code go?
            \end{itemize}
        \end{theorembox}
        \vspace{0.3em}
        {\small\color{TuebingenRot} \textit{Productivity gains of 20-40\% reported, but governance is non-negotiable.}}
    \end{columns}
\end{frame}

\begin{frame}{Editor Comparison: Features and Pricing}
    \framesubtitle{What do you get, and at what cost?}
    \begin{center}
    \begin{tabular}{|l|l|l|l|}
        \hline
        	extbf{Editor} & \textbf{Key Features} & \textbf{AI Integration} & \textbf{Pricing} \\
        \hline
        VS Code & Extensible, cross-platform & Copilot, 3rd-party & Free \\
        Cursor & AI-native, context tools & Built-in, advanced & Free basic, Paid Pro (\euro20+/mo) \\
        JetBrains & Refactoring, static analysis & AI Assistant (add-on) & \euro20-50/mo/user \\
        Vim/Neovim & Lightweight, scriptable & Plugins (Copilot, etc.) & Free \\
        \hline
    \end{tabular}
    \end{center}
    \vspace{0.5em}
    {\small\color{TuebingenGray} *Prices as of 2026, may vary by region and plan.}
\end{frame}

% ==============================================================================
% 0.6 Database Systems Overview (2 slides)
% ==============================================================================

\begin{frame}{Database Systems: Choosing the Right Tool}
    \framesubtitle{Different data patterns require different systems}
    
    {\color{TuebingenGray} "Which database?" is really "What are your query patterns and constraints?"}
    
    \vspace{0.3em}
    \begin{columns}[T]
        \column{0.48\textwidth}
        \textbf{\color{TuebingenRot}Relational (SQL):}
        \begin{itemize}
            \setlength\itemsep{0.2em}
            \item PostgreSQL, MySQL, Oracle
            \item Transactions, consistency, reporting
            \item Structured business data
            \item \textit{Most enterprise use cases}
        \end{itemize}
        
        \vspace{0.3em}
        \textbf{\color{TuebingenGold}Document Stores:}
        \begin{itemize}
            \setlength\itemsep{0.2em}
            \item MongoDB, CouchDB
            \item Flexible schemas
            \item Product data, events, logs
        \end{itemize}
        
        \vspace{0.3em}
        \textbf{\color{TuebingenGreen}Key-Value Stores:}
        \begin{itemize}
            \setlength\itemsep{0.2em}
            \item Redis, DynamoDB
            \item Caching, session state
            \item Sub-millisecond latency
        \end{itemize}
        
        \column{0.48\textwidth}
        \textbf{\color{TuebingenCyan}Columnar / Data Warehouses:}
        \begin{itemize}
            \setlength\itemsep{0.2em}
            \item Snowflake, BigQuery, Redshift
            \item Analytics at scale
            \item Historical analysis, BI
        \end{itemize}
        
        \vspace{0.3em}
        \textbf{\color{TuebingenAnthrazit}Graph Databases:}
        \begin{itemize}
            \setlength\itemsep{0.2em}
            \item Neo4j, Amazon Neptune
            \item Relationships, knowledge graphs
            \item Entity resolution, networks
        \end{itemize}
        
        \vspace{0.3em}
        \textbf{\color{TuebingenRot}Vector Databases:}
        \begin{itemize}
            \setlength\itemsep{0.2em}
            \item Pinecone, Weaviate, Qdrant
            \item \textbf{Embeddings for RAG/AI}
            \item Similarity search
        \end{itemize}
    \end{columns}
\end{frame}

\begin{frame}{Database Selection: Executive Decision Framework}
    \framesubtitle{Match the system to your constraints}
    
    \vspace{0.5em}
    \begin{center}
        \begin{tikzpicture}[
            node distance=1.5cm,
            box/.style={draw=TuebingenAnthrazit, rounded corners=3pt, minimum width=3cm, minimum height=0.8cm, align=center, font=\small, fill=TuebingenBeige},
            decision/.style={draw=TuebingenRot, rounded corners=3pt, minimum width=2.5cm, minimum height=0.6cm, align=center, font=\scriptsize, fill=TuebingenRot!10},
            arrow/.style={->, thick, TuebingenAnthrazit}
        ]
            \node[decision] (q1) at (0,0) {Need ACID transactions?};
            \node[box] (sql) at (5,0) {\textbf{SQL Database}\\PostgreSQL, MySQL};
            \node[decision] (q2) at (0,-1.5) {Analytics at scale?};
            \node[box] (warehouse) at (5,-1.5) {\textbf{Data Warehouse}\\Snowflake, BigQuery};
            \node[decision] (q3) at (0,-3) {AI similarity search?};
            \node[box] (vector) at (5,-3) {\textbf{Vector DB}\\Pinecone, Weaviate};
            \node[decision] (q4) at (0,-4.5) {Complex relationships?};
            \node[box] (graph) at (5,-4.5) {\textbf{Graph DB}\\Neo4j};
            
            \draw[arrow] (q1) -- node[above]{\tiny Yes} (sql);
            \draw[arrow] (q1) -- node[left]{\tiny No} (q2);
            \draw[arrow] (q2) -- node[above]{\tiny Yes} (warehouse);
            \draw[arrow] (q2) -- node[left]{\tiny No} (q3);
            \draw[arrow] (q3) -- node[above]{\tiny Yes} (vector);
            \draw[arrow] (q3) -- node[left]{\tiny No} (q4);
            \draw[arrow] (q4) -- node[above]{\tiny Yes} (graph);
        \end{tikzpicture}
    \end{center}
    
    \vspace{0.5em}
    \begin{theorembox}{Executive Rule of Thumb}
        Choose databases based on \textbf{query patterns} and \textbf{constraints} (latency, consistency, scale)—not vendor marketing or "fashion". Most organizations need \textbf{multiple} database types working together.
    \end{theorembox}
\end{frame}

% ==============================================================================
% 0.7 Compute Basics: CPU vs GPU (1 slide)

% ------------------------------------------------------------------------------
\begin{frame}{Compute Infrastructure: CPU, GPU, and Beyond}
    \framesubtitle{Understanding what powers AI workloads}
    {\color{TuebingenGray} AI's compute demands are fundamentally different from traditional software.}
    \vspace{0.5em}
    \begin{columns}[T]
        \column{0.32\textwidth}
        \begin{center}
            {\Large \color{TuebingenRot} \textbf{CPU}}
        \end{center}
        \vspace{0.2em}
        \begin{itemize}
            \setlength\itemsep{0.2em}
            \item General-purpose processor for all software
            \item Handles complex logic, branching, and orchestration
            \item Low to moderate parallelism (8-64 cores typical)
            \item Used for data processing, serving, business logic
            \item Widely available, cost-effective
        \end{itemize}
        \vspace{0.3em}
        {\scriptsize\color{TuebingenGray} \textit{Best for: Traditional software, ETL, web servers, orchestration}}
        \column{0.32\textwidth}
        \begin{center}
            {\Large \color{TuebingenGold} \textbf{GPU}}
        \end{center}
        \vspace{0.2em}
        \begin{itemize}
            \setlength\itemsep{0.2em}
            \item Massively parallel (10,000+ cores)
            \item Optimized for matrix math, deep learning
            \item Essential for AI model training and fast inference
            \item High memory bandwidth, but expensive
            \item Used in cloud and on-prem for ML/AI
        \end{itemize}
        \vspace{0.3em}
        {\scriptsize\color{TuebingenGray} \textit{Best for: Deep learning, large-scale analytics, simulation}}
        \column{0.32\textwidth}
        \begin{center}
            {\Large \color{TuebingenGreen} \textbf{TPU/NPU}}
        \end{center}
        \vspace{0.2em}
        \begin{itemize}
            \setlength\itemsep{0.2em}
            \item AI-specialized silicon (Google TPU, Apple NPU, etc.)
            \item Even more efficient for neural nets
            \item Used in cloud (TPU) and edge/mobile (NPU)
            \item Vendor lock-in risk, less flexible
            \item Can dramatically reduce inference cost at scale
        \end{itemize}
        \vspace{0.3em}
        {\scriptsize\color{TuebingenGray} \textit{Best for: Cloud AI services, edge inference, mobile AI}}
    \end{columns}
    \vspace{0.8em}
    \begin{columns}[T]
        \column{0.65\textwidth}
        	extbf{\color{TuebingenRot}On "Quantum Computing":}
        \begin{itemize}
            \setlength\itemsep{0.2em}
            \item \textbf{Not relevant} for mainstream ML today
            \item Potential future niche: optimization, simulation
            \item Separate timeline from current AI ROI discussions
        \end{itemize}
        \column{0.32\textwidth}
        {\small\color{TuebingenAnthrazit}
        	extbf{Cost Reality:}\\
        GPU compute is expensive.\\
        H100: \$25-50/hour (cloud)\\
        Training GPT-4: \$50-100M+\\
        CPU: \$0.05-0.20/hour\\
        TPU: \$10-30/hour (cloud)
        }
    \end{columns}
\end{frame}

\begin{frame}{Compute in Practice: Cost and Selection}
    \framesubtitle{How to choose and what to expect}
    \begin{itemize}
        \item \textbf{Cost drivers:} Model size, training duration, inference volume, hardware type, cloud vs. on-prem
        \item \textbf{Cloud pricing (2026):}
        \begin{itemize}
            \item CPU: \$0.05–0.20/hour (AWS, Azure, GCP)
            \item GPU (NVIDIA H100): \$25–50/hour
            \item TPU: \$10–30/hour
        \end{itemize}
        \item \textbf{Example:} Training a large LLM (GPT-4 scale) can cost \$50–100M+ in compute alone
        \item \textbf{Inference:} Serving a single query can cost \$0.001–0.01 (GPU), much less on CPU for small models
        \item \textbf{Selection guide:}
        \begin{itemize}
            \item \textbf{CPU:} Use for traditional software, small models, orchestration
            \item \textbf{GPU:} Use for deep learning, large models, fast inference
            \item \textbf{TPU/NPU:} Use for specialized AI, edge/mobile, or when vendor lock-in is acceptable
        \end{itemize}
        \item \textbf{Tip:} Start with cloud GPUs for flexibility, optimize for cost as usage grows
    \end{itemize}
    \vspace{0.5em}
    {\small\color{TuebingenGray} Always monitor usage and optimize for your workload.}
\end{frame}
