% Act 0 — Software Literacy Primer
\section{Orientation: Software Foundations Executives Need}

% ==============================================================================
% 0.1 The "Stack Map" (1 slide)
% ==============================================================================

\begin{frame}{The Technology Stack: AI in Context}
    \framesubtitle{Understanding where AI fits in your organization}
    
    {\color{TuebingenGray} AI is not magic—it's software built on data, running on infrastructure, serving business processes.}
    
    \vspace{0.2em}
    \begin{columns}[T,onlytextwidth]
        \column{0.44\textwidth}
        % Scaled down TikZ graph on the left
        \begin{center}
        \begin{tikzpicture}[scale=0.95, every node/.style={transform shape},
            box/.style={draw=TuebingenAnthrazit, rounded corners=3pt, minimum width=2.2cm, minimum height=0.7cm, align=center, font=\scriptsize},
            arrow/.style={->, thick, TuebingenGold}
        ]
            % Stack layers (bottom to top, closer together), all moved up by 0.8
            \node[box, fill=TuebingenBeige] (infra) at (0,0.9) {Infrastructure\\{\tiny CPU/GPU, Cloud}};
            \node[box, fill=TuebingenBeige] (data) at (0,1.85) {Data Systems\\{\tiny Databases, Pipelines}};
            \node[box, fill=TuebingenBeige] (software) at (0,2.8) {Software Systems\\{\tiny APIs, Services}};
            \node[box, fill=TuebingenRot!20] (ai) at (0,3.75) {ML/AI Models\\{\tiny Training, Inference}};
            \node[box, fill=TuebingenBeige] (product) at (0,4.7) {Product Interface\\{\tiny Apps, Dashboards}};
            \node[box, fill=TuebingenBeige] (govern) at (0,5.65) {Monitoring \& Governance\\{\tiny Compliance, Ops}};
            % Arrows
            \draw[arrow] (infra) -- (data);
            \draw[arrow] (data) -- (software);
            \draw[arrow] (software) -- (ai);
            \draw[arrow] (ai) -- (product);
            \draw[arrow] (product) -- (govern);
        \end{tikzpicture}
        \end{center}

        \column{0.53\textwidth}
            % Business context
            \vspace{0.1em}
            \setstretch{1.12}
            \textbf{Executive Reality:}\\[0.25em]
            \scriptsize
            $\blacktriangleright$ AI requires \textit{all} layers working\\[0.18em]
            $\blacktriangleright$ Model is often < 20\% of effort\\[0.18em]
            $\blacktriangleright$ Data quality gates success\\[0.18em]
            $\blacktriangleright$ Governance is not optional
            \
            \vspace{1.1em}
            
            \small\color{TuebingenRot} \textbf{Key insight:} "AI" is software + data + evaluation. Invest across the stack.
    \end{columns}
\end{frame}

% ==============================================================================
% 0.2 Programming Languages: What Exists and Why It Matters (2 slides)
% ==============================================================================

\begin{frame}{Programming Languages: The Landscape}
    \framesubtitle{Why language choice matters for your AI initiatives}
    
    {\color{TuebingenGray} Different languages serve different purposes. Understanding this helps evaluate team composition and vendor choices.}
    
    \vspace{0.5em}
    \begin{columns}[T]
        \column{0.48\textwidth}
        \textbf{\color{TuebingenRot}Data \& ML Ecosystem:}
        \begin{itemize}
            \setlength\itemsep{0.3em}
            \item \textbf{Python} — dominant for ML/AI
                \begin{itemize}
                    \item Rich libraries (TensorFlow, PyTorch)
                    \item Rapid prototyping
                    \item Data science standard
                \end{itemize}
            \item \textbf{R / MATLAB} — statistical analysis niches
        \end{itemize}
        
        \vspace{0.5em}
        \textbf{\color{TuebingenGold}Enterprise \& Backend:}
        \begin{itemize}
            \setlength\itemsep{0.3em}
            \item \textbf{Java} — enterprise systems, stability
            \item \textbf{Go} — cloud infrastructure, concurrency
            \item \textbf{C\#} — Microsoft ecosystem
        \end{itemize}
        
        \column{0.48\textwidth}
        \textbf{\color{TuebingenGreen}Performance-Critical:}
        \begin{itemize}
            \setlength\itemsep{0.3em}
            \item \textbf{C/C++} — model runtimes, systems
            \item \textbf{Rust} — safety + performance
            \item \textbf{CUDA} — GPU programming
        \end{itemize}
        
        \vspace{0.5em}
        \textbf{\color{TuebingenCyan}Product Interfaces:}
        \begin{itemize}
            \setlength\itemsep{0.3em}
            \item \textbf{JavaScript/TypeScript} — web, full-stack
            \item \textbf{Swift/Kotlin} — mobile apps
        \end{itemize}
        
        \vspace{0.5em}
        \textbf{\color{TuebingenAnthrazit}Specialized:}
        \begin{itemize}
            \setlength\itemsep{0.3em}
            \item \textbf{Haskell/Scala} — type safety, correctness
            \item \textbf{SQL} — data querying (ubiquitous)
        \end{itemize}
    \end{columns}
\end{frame}

\begin{frame}{Why Language Choice Matters for AI}
    \framesubtitle{Ecosystems, talent, and AI assistance quality}
    
    \vspace{0.5em}
    \begin{columns}[T]
        \column{0.55\textwidth}
        \textbf{The "Gravity Well" Effect:}
        \begin{itemize}
            \setlength\itemsep{0.5em}
            \item ML research concentrates in \textbf{Python}
            \item Enterprise gravity in \textbf{Java/Go}
            \item Performance work in \textbf{C++/Rust}
            \item Each ecosystem has its own:
                \begin{itemize}
                    \item Package libraries
                    \item Community expertise
                    \item Hiring pool
                \end{itemize}
        \end{itemize}
        
        \column{0.42\textwidth}
        \begin{theorembox}{AI Coding Assistant Quality}
            LLM coding tools perform best where \textbf{training data is abundant}.
            
            \vspace{0.3em}
            {\small
            \textbf{Strong support:} Python, JavaScript, Java, Go
            
            \textbf{Moderate:} C++, Rust, TypeScript
            
            \textbf{Weaker:} MATLAB, R, niche languages}
        \end{theorembox}
    \end{columns}
    
    \vspace{0.8em}
    \begin{center}
        \small\color{TuebingenRot} \textbf{Executive takeaway:} Language choice shapes experimentation speed, maintainability, hiring, and AI-assistance leverage.
    \end{center}
\end{frame}

% ==============================================================================
% 0.3 One Identical Example in Three Languages (3 slides)
% ==============================================================================

\begin{frame}[fragile]{Code Comparison: The Same Task in Three Languages}
    \framesubtitle{Task: Load CSV, compute summary, detect anomalies, output JSON}
    
    {\color{TuebingenGray} Seeing the same logic expressed differently reveals language philosophies.}
    
    \vspace{0.5em}
    \begin{columns}[T]
        \column{0.48\textwidth}
        \begin{codebox}
            {\small\textbf{\color{TuebingenGreen}Python} — Concise, library-rich}
            \vspace{0.3em}
\begin{lstlisting}[style=pythonstyle, basicstyle=\ttfamily\tiny]
import pandas as pd
import json

# Load and analyze
df = pd.read_csv("transactions.csv")
summary = {
    "total": df["amount"].sum(),
    "mean": df["amount"].mean(),
    "count": len(df)
}

# Detect anomalies (simple rule)
threshold = summary["mean"] * 3
anomalies = df[df["amount"] > threshold]
summary["anomalies"] = len(anomalies)

# Output
with open("report.json", "w") as f:
    json.dump(summary, f)
\end{lstlisting}
        \end{codebox}
        
        \column{0.48\textwidth}
        {\small\color{TuebingenAnthrazit}
        \textbf{Characteristics:}
        \begin{itemize}
            \setlength\itemsep{0.2em}
            \item 15 lines of code
            \item Rich standard library
            \item Readable, minimal boilerplate
            \item Dynamic typing (flexible)
            \item Dominant in data science
        \end{itemize}
        
        \vspace{0.5em}
        \textbf{Trade-offs:}
        \begin{itemize}
            \setlength\itemsep{0.2em}
            \item Slower runtime than compiled
            \item Type errors found at runtime
            \item GIL limits parallelism
        \end{itemize}
        }
    \end{columns}
\end{frame}

\begin{frame}[fragile]{Code Comparison: Java — Enterprise Standard}
    \framesubtitle{Same task: More structure, explicit types, verbose}
    
    \begin{columns}[T]
        \column{0.55\textwidth}
        \begin{codebox}
            {\small\textbf{\color{TuebingenGreen}Java} — Explicit, structured}
            \vspace{0.2em}
\begin{lstlisting}[style=javastyle, basicstyle=\ttfamily\tiny]
public class TransactionAnalyzer {
    public static void main(String[] args) {
        List<Transaction> txns = loadCSV("transactions.csv");
        
        double total = txns.stream()
            .mapToDouble(Transaction::getAmount)
            .sum();
        double mean = total / txns.size();
        double threshold = mean * 3;
        
        long anomalyCount = txns.stream()
            .filter(t -> t.getAmount() > threshold)
            .count();
        
        Summary summary = new Summary(
            total, mean, txns.size(), anomalyCount);
        
        ObjectMapper mapper = new ObjectMapper();
        mapper.writeValue(
            new File("report.json"), summary);
    }
}
\end{lstlisting}
        \end{codebox}
        
        \column{0.42\textwidth}
        {\small\color{TuebingenAnthrazit}
        \textbf{Characteristics:}
        \begin{itemize}
            \setlength\itemsep{0.2em}
            \item ~25 lines (plus class definitions)
            \item Static typing (compile-time safety)
            \item Explicit structure
            \item Enterprise conventions
            \item Long-lived, maintainable codebases
        \end{itemize}
        
        \vspace{0.5em}
        \textbf{Trade-offs:}
        \begin{itemize}
            \setlength\itemsep{0.2em}
            \item More boilerplate
            \item Slower iteration
            \item Steeper learning curve
        \end{itemize}
        }
    \end{columns}
\end{frame}

\begin{frame}[fragile]{Code Comparison: Go — Modern Systems Language}
    \framesubtitle{Same task: Explicit error handling, built for services}
    

    \begin{columns}[T]
        \column{0.55\textwidth}
        \vspace{-0.7em} % Move code box further up in its column
        \begin{codebox}
\begin{lstlisting}[style=gostyle, basicstyle=\ttfamily\tiny]
func analyzeTransactions() error {
    file, err := os.Open("transactions.csv")
    if err != nil {
        return fmt.Errorf("open: %w", err)
    }
    defer file.Close()
    
    txns, err := parseCSV(file)
    if err != nil {
        return fmt.Errorf("parse: %w", err)
    }
    
    var total float64
    for _, t := range txns {
        total += t.Amount
    }
    mean := total / float64(len(txns))
    threshold := mean * 3
    
    var anomalies int
    for _, t := range txns {
        if t.Amount > threshold {
            anomalies++
        }
    }
    
    summary := Summary{Total: total, Mean: mean,
        Count: len(txns), Anomalies: anomalies}
    return writeJSON("report.json", summary)
}
\end{lstlisting}
        \end{codebox}
        
        \column{0.42\textwidth}
        {\small\color{TuebingenAnthrazit}
        \textbf{Characteristics:}
        \begin{itemize}
            \setlength\itemsep{0.2em}
            \item ~30 lines
            \item \textbf{Explicit error handling}
            \item Compiled, fast execution
            \item Built-in concurrency
            \item Cloud/DevOps standard
        \end{itemize}
        
        \vspace{0.5em}
        \textbf{Go Philosophy:}
        \begin{itemize}
            \setlength\itemsep{0.2em}
            \item "Errors are values"
            \item Simplicity over cleverness
            \item Designed for services
        \end{itemize}
        
        \vspace{0.5em}
        \color{TuebingenRot}\small\textit{Used by: Docker, Kubernetes, most cloud infrastructure}
        }
    \end{columns}
\end{frame}

% ==============================================================================
% 0.5 Editors and AI Coding Copilots (1 slide)

% ------------------------------------------------------------------------------

% --- Slide 1: Popular Editors Overview ---
\begin{frame}{Popular Code Editors for AI Development}
    \framesubtitle{Where modern software is written}
    {\color{TuebingenGray} Modern editors bridge AI and developers—shaping productivity and delivery.}
    \vspace{0.5em}
    \begin{itemize}
        \setlength\itemsep{0.6em}
        \item \textbf{VS Code:} Free, open source, huge extension ecosystem, strong AI integration (Copilot, 3rd-party), cross-platform.
        \item \textbf{Cursor:} AI-native fork of VS Code, built-in copilots, powerful context navigation, advanced AI plans (paid).
        \item \textbf{JetBrains Suite:} (IntelliJ, PyCharm, etc.) Paid, advanced refactoring, deep language support, enterprise features, strong static analysis, AI assistant (paid add-on).
        \item \textbf{Vim/Neovim:} Free, highly customizable, keyboard-driven, used by power users and on servers; AI plugin ecosystem.
    \end{itemize}
    \vspace{0.5em}
    {\small\color{TuebingenAnthrazit}
    \textit{Select the editor based on project needs, team skills, and security policy.}
    }
\end{frame}

% --- Slide 2: Capabilities of AI Coding Assistants ---
\begin{frame}{AI Coding Assistant Capabilities}
    \framesubtitle{How AI accelerates software engineering}
    \begin{itemize}
        \setlength\itemsep{0.6em}
        \item \textbf{Code completion} (lines or entire blocks)
        \item \textbf{Refactoring suggestions}
        \item \textbf{Test case and documentation generation}
        \item \textbf{Code search and explanation}
        \item \textbf{Automated bug fixing}
        \item \textbf{Agentic workflows:} Multi-step code changes
    \end{itemize}
    \vspace{0.9em}
    {\small\color{TuebingenGray}
    \textit{Copilots are your co-engineers—but still need oversight.}
    }
\end{frame}

% --- Slide 3: Governance & Risk Implications for AI-Assisted Coding ---
\begin{frame}{Governance for AI Coding Tools}
    \framesubtitle{Critical issues for leadership and IT security}
    \begin{theorembox}{Governance Implications}
        \begin{itemize}
            \setlength\itemsep{0.4em}
            \item \textbf{Secrets} — Prevent AI from leaking credentials or API keys
            \item \textbf{IP/Licensing} — Know what's in AI training data; clarify code ownership
            \item \textbf{Security} — Beware of insecure or vulnerable AI-generated code
            \item \textbf{Auditability} — Maintain records: Who authored and reviewed code?
            \item \textbf{Data residency} — Ensure compliance: Is company code/data sent to outside servers?
        \end{itemize}
    \end{theorembox}
    \vspace{0.4em}
    {\small\color{TuebingenRot} \textit{Productivity gains can be 20--40\%, but governance is non-negotiable.}}
\end{frame}

% --- Slide 4: Editor Comparison Table ---
\begin{frame}{Editor Comparison: Key Features and Pricing}
    \framesubtitle{Weighing options for your organization}
    \begin{center}
    \renewcommand{\arraystretch}{1.13}
    \setlength{\tabcolsep}{6pt}
    \begin{tabular}{@{}l c c c@{}}
        \toprule
        \textbf{Editor} & \textbf{Key Features} & \textbf{AI Integration} & \textbf{Pricing} \\
        \midrule
        \textbf{VS Code}       & Extensible, cross-platform        & Copilot, 3rd-party  & Free \\
        \textbf{Cursor}        & AI-native, deep context tools     & Built-in (advanced) & Free Basic,\\
                              &                                  &                     & {\footnotesize Paid Pro (\euro20+/mo)} \\
        \textbf{JetBrains}     & Refactoring, static analysis      & AI Assistant (add-on) & \euro20--50/mo/user \\
        \textbf{Vim/Neovim}    & Lightweight, scriptable           & Plugins (Copilot, etc.) & Free \\
        \bottomrule
    \end{tabular}
    \end{center}
    \vspace{0.6em}
    {\tiny\color{TuebingenGray} Prices as of 2026. Check vendors for updates; enterprise plans may differ.}
\end{frame}

% ==============================================================================
% 0.6 Database Systems Overview (8 slides)
% ==============================================================================

% --- Slide 1: Database Overview ---
\begin{frame}{Database Systems: The Foundation of Data-Driven Business}
    \framesubtitle{Why database choice determines your AI success}
    
    {\color{TuebingenGray} Your database architecture is the foundation that enables or constrains every AI initiative.}
    
    \vspace{0.7em}
    \begin{itemize}
        \item Databases store and organize data essential for all modern software and AI applications
        \item The right database enables scalability, reliability, and efficient data access
        \item Poor database choices can severely limit future AI initiatives and analytics
        \item Evaluate based on your needs: access speed, consistency, scalability, and security
    \end{itemize}
    
    \vspace{0.5em}
    \begin{theorembox}{Executive Reality}
        Database decisions made today will impact your organization for 5-10 years. Choose based on \textbf{access patterns}, \textbf{consistency requirements}, and \textbf{scale projections}—not vendor relationships.
    \end{theorembox}
\end{frame}

% --- Slide 2: Relational Databases (SQL) ---
\begin{frame}{Relational Databases: The Enterprise Backbone}
    \framesubtitle{PostgreSQL, MySQL, Oracle, SQL Server}
    
    \vspace{0.1em}
    \begin{columns}[T]
        \column{0.45\textwidth}
        \begin{center}
            \begin{tikzpicture}[scale=0.95]
                % Table representation
                \draw[thick, TuebingenAnthrazit] (0,0) rectangle (3.5,2.5);
                \draw[thick, TuebingenAnthrazit] (0,2) -- (3.5,2);
                \draw[thick, TuebingenAnthrazit] (0,1.5) -- (3.5,1.5);
                \draw[thick, TuebingenAnthrazit] (0,1) -- (3.5,1);
                \draw[thick, TuebingenAnthrazit] (0,0.5) -- (3.5,0.5);
                \draw[thick, TuebingenAnthrazit] (1.2,0) -- (1.2,2.5);
                \draw[thick, TuebingenAnthrazit] (2.4,0) -- (2.4,2.5);
                
                % Headers
                \node[font=\tiny\bfseries] at (0.6,2.25) {ID};
                \node[font=\tiny\bfseries] at (1.8,2.25) {Name};
                \node[font=\tiny\bfseries] at (2.95,2.25) {Email};
                
                % Data rows
                \node[font=\tiny] at (0.6,1.75) {1};
                \node[font=\tiny] at (1.8,1.75) {Alice};
                \node[font=\tiny] at (2.95,1.75) {a@co.com};
                
                \node[font=\tiny] at (0.6,1.25) {2};
                \node[font=\tiny] at (1.8,1.25) {Bob};
                \node[font=\tiny] at (2.95,1.25) {b@co.com};
                
                \node[font=\tiny] at (0.6,0.75) {3};
                \node[font=\tiny] at (1.8,0.75) {Carol};
                \node[font=\tiny] at (2.95,0.75) {c@co.com};
                
                % Title
                \node[above, font=\scriptsize\bfseries, color=TuebingenRot] at (1.75,2.5) {Structured Tables};
            \end{tikzpicture}
        \end{center}
        
        \column{0.52\textwidth}
        \textbf{\color{TuebingenRot}Key Characteristics:}
        \begin{itemize}
            \setlength\itemsep{0.3em}
            \item \textbf{ACID Transactions} — Guaranteed consistency
            \item \textbf{SQL Query Language} — Standardized, powerful
            \item \textbf{Referential Integrity} — Enforced relationships
            \item \textbf{Mature Ecosystem} — Tools, expertise, support
        \end{itemize}
    \end{columns}
    
    \vspace{0.4em}
    \textbf{\color{TuebingenGold}Best For:}
    \begin{itemize}
        \setlength\itemsep{0.2em}
        \item Financial transactions, customer records, inventory
        \item Complex reporting and business intelligence
        \item Applications requiring strong consistency
        \item Most traditional enterprise workloads
    \end{itemize}
    
    \vspace{0.3em}
    {\small\color{TuebingenGray} \textit{Rule of thumb: Start with PostgreSQL unless you have specific reasons not to.}}
\end{frame}

% --- Slide 3: Document Databases ---
\begin{frame}{Document Databases: Flexible Schema for Modern Apps}
    \framesubtitle{MongoDB, CouchDB, Amazon DocumentDB}
    
    \begin{columns}[T]
        \column{0.45\textwidth}
        \begin{center}
            \begin{tikzpicture}[scale=1]
                % Document representation (bigger, centered in column
                % JSON-like structure (bigger font, properly aligned within rectangle)
                \node[font=\normalsize\ttfamily, align=left, anchor=north west] at (0.34,5.0) {
                    \textcolor{TuebingenRot}{\{} \\
                    \quad "id": "user123", \\
                    \quad "name": "Alice", \\
                    \quad "email": "a@co.com", \\
                    \quad "preferences": \{ \\
                    \quad\quad "theme": "dark", \\
                    \quad\quad "notifications": true \\
                    \quad \}, \\
                    \quad "tags": ["premium", "beta"] \\
                    \textcolor{TuebingenRot}{\}}
                };

                % Title above the document (bigger and centered above rectangle)
                \node[above, font=\large\bfseries, color=TuebingenGold] at (2.65,5.45) {JSON Documents};
            \end{tikzpicture}
        \end{center}
        
        \column{0.52\textwidth}
        \textbf{\color{TuebingenGold}Key Characteristics:}
        \begin{itemize}
            \setlength\itemsep{0.3em}
            \item \textbf{Schema Flexibility} — Add fields without migration
            \item \textbf{Nested Data} — Complex objects in single document
            \item \textbf{Horizontal Scaling} — Distribute across servers
            \item \textbf{Developer Friendly} — Maps to application objects
        \end{itemize}
    \end{columns}
    
    \vspace{0.3em}
    {\small\color{TuebingenGray} \textit{Trade-off: Flexibility vs. consistency guarantees. Great for read-heavy workloads.}}
\end{frame}

% --- Slide 5: Columnar/Data Warehouses ---
\begin{frame}{Data Warehouses: Analytics at Enterprise Scale}
    \framesubtitle{Snowflake, BigQuery, Redshift, ClickHouse}
    
    \vspace{0.3em}
    \begin{columns}[T]
        \column{0.45\textwidth}
        \begin{center}
            \begin{tikzpicture}[scale=0.7]
                % Columnar storage visualization
                \draw[thick, TuebingenAnthrazit] (0,0) rectangle (4,2.5);
                
                % Column divisions
                \draw[thick, TuebingenAnthrazit] (1,0) -- (1,2.5);
                \draw[thick, TuebingenAnthrazit] (2,0) -- (2,2.5);
                \draw[thick, TuebingenAnthrazit] (3,0) -- (3,2.5);
                
                % Column headers
                \node[font=\tiny\bfseries, color=TuebingenCyan] at (0.5,2.25) {Date};
                \node[font=\tiny\bfseries, color=TuebingenCyan] at (1.5,2.25) {Product};
                \node[font=\tiny\bfseries, color=TuebingenCyan] at (2.5,2.25) {Sales};
                \node[font=\tiny\bfseries, color=TuebingenCyan] at (3.5,2.25) {Region};
                
                % Data (compressed columns)
                \foreach \i in {0,1,2,3,4} {
                    \fill[TuebingenCyan!30] (0,1.8-\i*0.3) rectangle (1,1.5-\i*0.3);
                    \fill[TuebingenGold!30] (1,1.8-\i*0.3) rectangle (2,1.5-\i*0.3);
                    \fill[TuebingenGreen!30] (2,1.8-\i*0.3) rectangle (3,1.5-\i*0.3);
                    \fill[TuebingenRot!30] (3,1.8-\i*0.3) rectangle (4,1.5-\i*0.3);
                }
                
                % Compression indicator
                \node[font=\tiny, color=TuebingenGray] at (2,-0.3) {Compressed Columns};
                
                % Title
                \node[above, font=\scriptsize\bfseries, color=TuebingenCyan] at (2,2.8) {Columnar Storage};
            \end{tikzpicture}
        \end{center}
        
        \column{0.52\textwidth}
        \textbf{\color{TuebingenCyan}Key Characteristics:}
        \begin{itemize}
            \setlength\itemsep{0.3em}
            \item \textbf{Columnar Storage} — Optimized for analytics
            \item \textbf{Massive Parallelism} — Query across clusters
            \item \textbf{Compression} — 10x+ storage efficiency
            \item \textbf{SQL Interface} — Familiar query language
        \end{itemize}
    \end{columns}
    
    \vspace{0.4em}
    \textbf{\color{TuebingenCyan}Best For:}
    \begin{itemize}
        \setlength\itemsep{0.2em}
        \item Business intelligence and reporting
        \item Data science and ML feature engineering
        \item Regulatory compliance and auditing
    \end{itemize}
    
    \vspace{0.3em}
    {\small\color{TuebingenGray} \textit{Essential for AI: Most ML features come from aggregating historical data.}}
\end{frame}

% --- Slide 6: Graph Databases ---
\begin{frame}{Graph Databases: Relationships as First-Class Citizens}
    \framesubtitle{Neo4j, Amazon Neptune, ArangoDB}
    
    \vspace{0.3em}
    \begin{columns}[T]
        \column{0.45\textwidth}
        \begin{center}
            \begin{tikzpicture}[scale=0.8]
                % Nodes
                \node[circle, draw=TuebingenAnthrazit, fill=TuebingenRot!20, minimum size=0.8cm, font=\tiny] (alice) at (0,2) {Alice};
                \node[circle, draw=TuebingenAnthrazit, fill=TuebingenGold!20, minimum size=0.8cm, font=\tiny] (bob) at (2,2) {Bob};
                \node[circle, draw=TuebingenAnthrazit, fill=TuebingenGreen!20, minimum size=0.8cm, font=\tiny] (carol) at (1,0.5) {Carol};
                \node[circle, draw=TuebingenAnthrazit, fill=TuebingenCyan!20, minimum size=0.8cm, font=\tiny] (dave) at (3,0.5) {Dave};
                
                % Relationships
                \draw[->, thick, TuebingenAnthrazit] (alice) -- (bob) node[midway, above, font=\tiny] {KNOWS};
                \draw[->, thick, TuebingenAnthrazit] (bob) -- (carol) node[midway, right, font=\tiny] {WORKS\_WITH};
                \draw[->, thick, TuebingenAnthrazit] (carol) -- (dave) node[midway, below, font=\tiny] {MANAGES};
                \draw[->, thick, TuebingenAnthrazit] (alice) -- (carol) node[midway, left, font=\tiny] {REPORTS\_TO};
                \draw[->, thick, TuebingenAnthrazit] (bob) -- (dave) node[midway, above right, font=\tiny] {COLLABORATES};
                
                % Title
                \node[above, font=\scriptsize\bfseries, color=TuebingenAnthrazit] at (1.5,2.8) {Connected Data};
            \end{tikzpicture}
        \end{center}
        
        \column{0.52\textwidth}
        \textbf{\color{TuebingenAnthrazit}Key Characteristics:}
        \begin{itemize}
            \setlength\itemsep{0.3em}
            \item \textbf{Native Relationships} — Connections are data
            \item \textbf{Graph Traversal} — Follow paths efficiently
            \item \textbf{Pattern Matching} — Find complex relationships
            \item \textbf{Real-time Queries} — Interactive exploration
        \end{itemize}
    \end{columns}
    
    \vspace{0.4em}
    \textbf{\color{TuebingenAnthrazit}Best For:}
    \begin{itemize}
        \setlength\itemsep{0.2em}
        \item Social networks and recommendation engines
        \item Knowledge graphs and semantic search
        \item Supply chain and network analysis
    \end{itemize}
    
    \vspace{0.3em}
    {\small\color{TuebingenGray} \textit{When relationships are as important as the entities themselves.}}
\end{frame}

% --- Slide 7: Vector Databases ---
\begin{frame}{Vector Databases: The AI-Native Data Store}
    \framesubtitle{Pinecone, Weaviate, Qdrant, Chroma}
    
    \vspace{0.3em}
    \begin{columns}[T]
        \column{0.45\textwidth}
        \begin{center}
            \begin{tikzpicture}[scale=0.8]
                % Vector space representation
                \draw[->] (0,0) -- (3.5,0) node[right] {\tiny dim 1};
                \draw[->] (0,0) -- (0,3) node[above] {\tiny dim 2};
                
                % Vector points
                \fill[TuebingenRot] (1,1.5) circle (2pt);
                \fill[TuebingenGold] (2.5,2) circle (2pt);
                \fill[TuebingenGreen] (1.8,0.8) circle (2pt);
                \fill[TuebingenCyan] (0.8,2.2) circle (2pt);
                
                % Labels
                \node[font=\tiny, right] at (1,1.5) {doc1};
                \node[font=\tiny, right] at (2.5,2) {doc2};
                \node[font=\tiny, right] at (1.8,0.8) {doc3};
                \node[font=\tiny, right] at (0.8,2.2) {query};
                
                % Similarity circle
                \draw[dashed, TuebingenCyan] (0.8,2.2) circle (0.7);
                \node[font=\tiny, color=TuebingenCyan] at (0.2,2.8) {similar};
                
                % Title
                \node[above, font=\scriptsize\bfseries, color=TuebingenRot] at (1.75,3.3) {Embedding Space};
            \end{tikzpicture}
        \end{center}
        
        \column{0.52\textwidth}
        \textbf{\color{TuebingenRot}Key Characteristics:}
        \begin{itemize}
            \setlength\itemsep{0.3em}
            \item \textbf{High-Dimensional Vectors} — 100s to 1000s of dimensions
            \item \textbf{Similarity Search} — Find "nearest neighbors"
            \item \textbf{AI-Native} — Built for embeddings
            \item \textbf{Hybrid Search} — Combine with metadata filtering
        \end{itemize}
    \end{columns}
    
    \vspace{0.4em}
    \textbf{\color{TuebingenRot}Best For:}
    \begin{itemize}
        \setlength\itemsep{0.2em}
        \item \textbf{RAG Systems} — Retrieval-Augmented Generation
        \item Semantic search and document retrieval
        \item Image and video similarity search
    \end{itemize}
    
    \vspace{0.3em}
    {\small\color{TuebingenGray} \textit{Critical for modern AI: Where your LLM's context comes from.}}
\end{frame}

% --- Slide 8: Multi-Database Reality (Executive Summary) ---
\begin{frame}{Multi-Database Reality}
    \framesubtitle{Modern enterprises use \textbf{multiple databases} working together}
    
    \begin{theorembox}{Multi-Database Reality}
        Modern organizations rarely use a single type of database. Instead, they combine several specialized databases in one architecture:
        \begin{itemize}
            \item \textbf{Operational:} PostgreSQL or MySQL for core business data and transactions
            \item \textbf{Analytics:} Snowflake or BigQuery for large-scale business intelligence
            \item \textbf{Caching:} Redis for speed and responsiveness
            \item \textbf{AI:} Vector DBs (like Pinecone) for embeddings and search in RAG systems
        \end{itemize}
        \textbf{Key point:} There is no single "best" database. Use the right tool for each job, and design your stack to integrate these smoothly.
    \end{theorembox}
\end{frame}

% ==============================================================================
% 0.7 Compute Basics (4 slides)
% ------------------------------------------------------------------------------

% --- Slide 1: CPUs: Versatile Workhorse of Computing ---
\begin{frame}{CPUs: The Versatile Workhorse}
    \framesubtitle{What are CPUs good for?}
    \begin{itemize}
        \item The central processor in almost every computer (laptops, servers, phones)
        \item Designed to handle a wide variety of tasks, often switching rapidly between them
        \item Great at running business software, web servers, databases, and orchestrating complex workflows
        \item Ubiquitous, reliable, and affordable
        \item \textbf{Example workloads:}\\
        \quad -- Traditional web backends \\
        \quad -- Office applications and business logic\\
        \quad -- Light machine learning inference (small models)\\
        \quad -- Scientific codes with lots of branching
    \end{itemize}
    \vspace{0.5em}
    {\small\color{TuebingenGray} \textit{Use CPUs for most software and orchestration — backbone of IT and engineering.}}
\end{frame}

% --- Slide 2: GPUs: Parallel Power for Simulations and AI ---
\begin{frame}{GPUs: Parallel Power for Simulations and AI}
    \framesubtitle{Why GPUs matter and where they shine}
    \begin{itemize}
        \item Originally built for graphics/cards for games and 3D rendering
        \item Excel at parallel processing: thousands of small cores run the same instruction on lots of data
        \item \textbf{Game-changer for AI and scientific computing}\\
        \item Much faster than CPUs for highly parallel problems
        \item Often accessed via cloud due to high cost
        \item \textbf{Example workloads:}\\
        \quad -- Large Language Models (LLMs), deep learning\\
        \quad -- Molecular dynamics (MD) simulation\\
        \quad -- Finite Element Method (FEM) for engineering\\
        \quad -- Video rendering, image recognition, big-data analytics
    \end{itemize}
    \vspace{0.5em}
    {\small\color{TuebingenGray} \textit{Use GPUs for modern ML, simulations, and number-crunching.}}
\end{frame}

% --- Slide 3: TPUs, NPUs, and Specialized AI Chips ---
\begin{frame}{TPUs, NPUs, and AI-Specific Hardware}
    \framesubtitle{Purpose-built for high-efficiency machine learning}
    \begin{itemize}
        \item \textbf{TPU}: "Tensor Processing Unit" (Google) — optimized for deep learning, especially in their cloud
        \item \textbf{NPU}: "Neural Processing Unit" — found in mobile devices for on-device AI
        \item Highly efficient for specific AI models (e.g., transformer inference, image recognition)
        \item Often less flexible than GPU, but faster for certain model types; cost can be lower per operation
        \item \textbf{Example workloads:}\\
        \quad -- Image classification on smartphones\\
        \quad -- “Smart” camera features (object detection, AR)\\
        \quad -- Large model inference (TPU for LLM deployment)
    \end{itemize}
    \vspace{0.5em}
    {\small\color{TuebingenGray}\textit{Choose TPUs/NPUs when you want high-volume, low-cost inference (edge or cloud).}}
\end{frame}

% --- Slide 4: Quantum Computing ---
\begin{frame}{Quantum Computing: Hype vs. Reality}
    \framesubtitle{What is quantum really good for?}
    {\textbf{\color{TuebingenRot}Quantum Computing:}}
    \begin{itemize}
        \setlength\itemsep{0.3em}
        \item \textbf{Still experimental for most applications}
        \item Hype is high, but most business and AI workloads (LLMs, MD, FEM) \textbf{run on classical hardware}
        \item Quantum may eventually help: 
        \begin{itemize}
            \item Breaking cryptography 
            \item Simulating quantum chemistry/materials
            \item Special classes of optimization
        \end{itemize}
        \item No real impact on practical ML/AI/LLMs yet
        \item Interesting for research, not for production workloads
    \end{itemize}
    \vspace{0.7em}
    {\small\color{TuebingenGray} \textit{Summary: Quantum is promising, but not directly relevant to applied AI/ML engineering today.}}
\end{frame}

% --- Slide 5: Modern Compute Hardware & Cost Cheat Sheet ---
\begin{frame}{Compute Hardware \& Cost Cheat Sheet}
    \framesubtitle{Capabilities, Example Costs, and Use Cases}
    \begin{center}
    {\begingroup
    \renewcommand{\arraystretch}{1.0} % 80% of original 1.45
    \setlength{\tabcolsep}{9pt} % Make columns wider (from 4.5pt)
    \scalebox{0.9}{
    {\small % <-- Make font smaller for the table content
    \begin{tabular}{>{\raggedright\arraybackslash}p{3.1cm} >{\raggedright\arraybackslash}p{3.7cm} >{\raggedright\arraybackslash}p{3.7cm} >{\raggedright\arraybackslash}p{3.2cm}}
        \rowcolor{TuebingenBeige}
        \textbf{Hardware} & \textbf{Strengths} & \textbf{Example Uses} & \textbf{\color{TuebingenRot}Typical Cost\textsuperscript{*}} \\
        \hline
        \rowcolor{white}
        \textbf{CPU} & General-purpose, flexible, orchestrates most workloads & Web servers, business apps, small ML inference, scientific codes & \$0.05--0.20/hr \\
        \rowcolor{TuebingenGold!10}
        \textbf{GPU (A100/H100)} & Massively parallel, fast for AI/simulations, deep learning & LLM/AI training \& inference, scientific computing, analytics, 3D graphics & \$2--\$10/hr (A100)\\
        & & & \$25--\$50/hr (H100, cloud) \\
        \rowcolor{white}
        \textbf{TPU / NPU} & Specialized for ML, efficient on large inference or edge AI & Large model inference (cloud), on-device AI (phones, cameras) & \$10--\$30/hr (TPU v4) \\
        \rowcolor{TuebingenGold!10}
        \hline
        \multicolumn{4}{l}{\footnotesize \textsuperscript{*}\textcolor{TuebingenGray}{Cloud (on-demand, June 2024). Enterprise deals, preemptible, or low-priority compute can be cheaper.}} \\
        \multicolumn{4}{l}{\footnotesize \textcolor{TuebingenRot}{LLM Training (GPT-4 scale):} \$50--100M+} \\
    \end{tabular}
    } % end small
    }
    \endgroup}
    \end{center}
    \vspace{0.48em}
    {\small\color{TuebingenGold} \textit{Takeaway: Always match workload to the right hardware for performance \& cost.}}
\end{frame}


% ==============================================================================
% 0.8 Software Engineering Roles and Compensation (now split into concise slides)
% ==============================================================================

% --- Slide 1: The Software Engineering Talent Landscape ---
\begin{frame}{Software Engineering Roles: Overview}
    \framesubtitle{Who builds what?}
    
    {\color{TuebingenGray} Different roles bring different skills, capabilities, and costs. The right mix is key for project success, hiring, and budgeting.}
    
    \vspace{0.5em}
    \begin{columns}[T]
        \column{0.5\textwidth}
        \centering
        \begin{tikzpicture}[
            role/.style={draw=TuebingenAnthrazit, rounded corners=3pt, minimum width=2.2cm, minimum height=0.7cm, align=center, font=\scriptsize\bfseries},
            node distance=1.8cm
        ]
            \node[role, fill=TuebingenCyan!30] (frontend) {Frontend};
            \node[role, fill=TuebingenGreen!30, right of=frontend] (backend) {Backend};
            \node[role, fill=TuebingenGold!30, right of=backend] (fullstack) {Full Stack};
            \node[role, fill=TuebingenRot!30, below of=backend] (ai) {AI/ML};
            \node[role, fill=TuebingenAnthrazit!20, left of=ai] (data) {Data\\Science};
            \node[role, fill=TuebingenBeige, right of=ai] (devops) {DevOps/SRE};
        \end{tikzpicture}
        
        \column{0.47\textwidth}
        \vspace{0.2em}
        \textbf{\color{TuebingenRot}Role Categories:}
        \begin{itemize}
            \setlength\itemsep{0.3em}
            \item \textbf{Product Engineering:} Frontend, Backend, Full Stack
            \item \textbf{AI \& Data:} AI/ML Engineers, Data Scientists
            \item \textbf{Infrastructure:} DevOps, SRE, Platform Engineers
        \end{itemize}
        
        \vspace{0.5em}
        \small\color{TuebingenGold} \textbf{Key Insight:} Each role commands different market rates based on scarcity, complexity, and business impact.
    \end{columns}
\end{frame}

% --- Slide 1b: Junior, Senior, Lead: What Do These Levels Mean? ---
\begin{frame}{What is Junior, Senior, Lead?}
    \framesubtitle{Understanding experience levels in engineering}
    
    \vspace{0.3em}
    \begin{columns}[T]
        \column{0.48\textwidth}
        \begin{theorembox}{Junior (0--2 years)}
            \footnotesize
            \begin{itemize}
                \setlength\itemsep{0.2em}
                \item May have internships, needs mentoring
                \item Works on defined tasks
                \item Learning core skills and tech stack
            \end{itemize}
        \end{theorembox}
        
        \vspace{0.4em}
        \begin{theorembox}{Mid-Level (2--4 years)}
            \footnotesize
            \begin{itemize}
                \setlength\itemsep{0.2em}
                \item Gaining independence
                \item Owns small features
                \item Begins mentoring juniors
            \end{itemize}
        \end{theorembox}
        
        \column{0.48\textwidth}
        \begin{theorembox}{Senior (4+ years)}
            \footnotesize
            \begin{itemize}
                \setlength\itemsep{0.2em}
                \item Proven delivery on complex projects
                \item Works independently, designs systems
                \item Technical "go-to", code reviewer
            \end{itemize}
        \end{theorembox}
        
        \vspace{0.4em}
        \begin{theorembox}{Lead (6+ years)}
            \footnotesize
            \begin{itemize}
                \setlength\itemsep{0.2em}
                \item Guides technical direction
                \item Makes architectural decisions
                \item Balances coding with leadership
            \end{itemize}
        \end{theorembox}
    \end{columns}
    
\end{frame}

% --- Slide 2: Web Development Roles ---
\begin{frame}{Frontend, Backend, Full Stack}
    \framesubtitle{Web/Product Engineers (2026 compensation, Euro)}
    
    \vspace{0.2em}
    \begin{center}
    \renewcommand{\arraystretch}{1.2}
    \setlength{\tabcolsep}{7pt}
    \footnotesize
    \begin{tabular}{@{}>{\color{TuebingenAnthrazit}\bfseries}l p{2.8cm} p{2.8cm} p{2.5cm}@{}}
        \toprule
        \rowcolor{TuebingenBeige}
        \textbf{Role} & \textbf{Main Skills} & \textbf{Capabilities} & \textbf{Salary Range} \\
        \midrule
        \rowcolor{TuebingenCyan!10}
        \textbf{Frontend} & HTML, CSS, JS, React, Vue & Web/mobile UIs, user experience & \textbf{Jr:} 45--65K \\
        & & & \textbf{Sr:} 70--95K \\
        & & & \textbf{Lead:} 100--130K \\
        \midrule
        \rowcolor{TuebingenGreen!10}
        \textbf{Backend} & Java, Python, Go, Databases & APIs, server logic, data layer & \textbf{Jr:} 50--70K \\
        & & & \textbf{Sr:} 75--105K \\
        & & & \textbf{Lead:} 110--140K \\
        \midrule
        \rowcolor{TuebingenGold!10}
        \textbf{Full Stack} & Frontend + Backend, Integration & End-to-end features, MVPs & \textbf{Jr:} 55--75K \\
        & & & \textbf{Sr:} 80--110K \\
        & & & \textbf{Lead:} 115--150K \\
        \bottomrule
    \end{tabular}
    \end{center}
    
    \vspace{0.5em}
    \begin{columns}[T]
        \column{0.5\textwidth}
        \small\color{TuebingenGray} \textit{Full stack offers versatility but less deep specialization.}
        \column{0.5\textwidth}
        \small\color{TuebingenGray} \textit{Backend/Frontend provide domain expertise.}
    \end{columns}
\end{frame}

% --- Slide 3: AI/ML, Data Science, DevOps/SRE ---
\begin{frame}{AI/ML, Data Science, DevOps/SRE}
    \framesubtitle{Specialized roles (2026 compensation, Euro)}
    
    \vspace{0.2em}
    \begin{center}
    \renewcommand{\arraystretch}{1.2}
    \setlength{\tabcolsep}{7pt}
    \footnotesize
    \begin{tabular}{@{}>{\color{TuebingenAnthrazit}\bfseries}l p{2.8cm} p{2.8cm} p{2.5cm}@{}}
        \toprule
        \rowcolor{TuebingenBeige}
        \textbf{Role} & \textbf{Main Skills} & \textbf{Capabilities} & \textbf{Salary Range} \\
        \midrule
        \rowcolor{TuebingenRot!15}
        \textbf{AI/ML Engineer} & Python, PyTorch, Statistics, MLOps & Train/deploy models, AI systems & \textbf{Jr:} 70--90K \\
        & & & \textbf{Sr:} 100--140K \\
        & & & \textbf{Lead:} 150--200K+ \\
        \midrule
        \rowcolor{TuebingenAnthrazit!10}
        \textbf{Data Scientist} & Python, R, SQL, Visualization, Stats & Analysis, business insights, ML features & \textbf{Jr:} 60--80K \\
        & & & \textbf{Sr:} 85--120K \\
        & & & \textbf{Lead:} 125--160K \\
        \midrule
        \rowcolor{TuebingenBeige}
        \textbf{DevOps/SRE} & AWS, Kubernetes, CI/CD, Infrastructure & Deploy, automate, scale, reliability & \textbf{Jr:} 60--85K \\
        & & & \textbf{Sr:} 90--125K \\
        & & & \textbf{Lead:} 130--170K \\
        \bottomrule
    \end{tabular}
    \end{center}
    
    \vspace{0.5em}
    \begin{columns}[T]
        \column{0.5\textwidth}
        \small\color{TuebingenRot} \textbf{Premium:} AI/ML commands highest salaries due to scarcity.
        \column{0.5\textwidth}
        \small\color{TuebingenGold} \textbf{Critical:} DevOps essential for scale and reliability.
    \end{columns}
\end{frame}

% --- Slide 3b: Why are AI/ML Engineer Salaries so High? (Key Reasons) ---
\begin{frame}{Why do AI/ML Engineers Earn So Much?}
    \framesubtitle{Four Core Reasons}

    \vspace{0.5em}
    \begin{description}[align=left]
        \item[\color{TuebingenRot}\textbf{1.~Scarcity \& Demand}]
            \vspace{0.3em}
            \begin{itemize}
                \item Demand for AI/ML engineers exceeds supply globally.
                \item Highly qualified talent (AI theory + scalable systems) is rare.
            \end{itemize}
        \item[\color{TuebingenGold}\textbf{2.~Business Value}]
            \vspace{0.3em}
            \begin{itemize}
                \item AI/ML enables major advances: ChatGPT, fraud detection, hyper-personalization.
                \item Drives product innovation, automation, and defensible company advantage.
            \end{itemize}
        \item[\color{TuebingenGreen}\textbf{3.~High Skill Bar}]
            \vspace{0.3em}
            \begin{itemize}
                \item Requires math, statistics, software engineering, and MLOps/infrastructure.
                \item Talent must handle complexity at large scale.
            \end{itemize}
        \item[\color{TuebingenCyan}\textbf{4.~Impact \& Evolution}]
            \vspace{0.3em}
            \begin{itemize}
                \item Field evolves rapidly (LLMs, new architectures).
                \item One top engineer can replace/amplify dozens, but requires constant upskilling.
            \end{itemize}
    \end{description}
\end{frame}

% --- Slide 3c: Executive Perspective: The Leverage of the Best ---
\begin{frame}{Executive Reality: The Leverage of Elite AI Talent}
    \framesubtitle{Why organizations pay a premium}

    \vspace{0.4em}
    \begin{theorembox}{Executive Reality}
        In many organizations, \textbf{just a handful of exceptional AI/ML engineers} drive transformative business outcomes across entire product lines. A single great engineer can:
        \begin{itemize}
            \setlength\itemsep{0.4em}
            \item Automate what previously required dozens of traditional roles
            \item Unlock new products or revenue streams through AI innovation
            \item Enable massive cost savings, new features, and company-wide differentiation
            \item Continuously adapt to fast-evolving technology landscapes (e.g., LLMs, generative AI)
        \end{itemize}
        \textbf{This leverage is why top AI/ML engineers are so highly compensated---they can literally change the trajectory of entire businesses.}
    \end{theorembox}
\end{frame}

% --- Slide 4: When to Hire What? ---
\begin{frame}{Role Selection: Who and When?}
    \framesubtitle{Quick guidelines for hiring decisions}
    
    \vspace{0.2em}
    \begin{columns}[T]
        \column{0.48\textwidth}
        \textbf{\color{TuebingenRot}When to Hire:}
        \begin{itemize}
            \setlength\itemsep{0.3em}
            \item \textbf{Frontend:} User interfaces, web/mobile apps
            \item \textbf{Backend:} APIs, databases, business logic
            \item \textbf{Full Stack:} Startups, MVPs, rapid prototypes
            \item \textbf{AI/ML:} Model development, AI features, MLOps
            \item \textbf{Data Scientist:} Analysis, business insights, ML features
            \item \textbf{DevOps/SRE:} Deployment, infrastructure, scaling, reliability
        \end{itemize}
        
        \column{0.48\textwidth}
        \textbf{\color{TuebingenGold}Team Size Patterns:}
        \begin{itemize}
            \setlength\itemsep{0.3em}
            \item \textbf{Startup:} 2--3 full stack, 1 AI/ML, 1 DevOps
            \item \textbf{Scale-up:} Split front/back, +AI/ML, +data science
            \item \textbf{Enterprise:} Full specialization, dedicated teams per domain
        \end{itemize}
        
        \vspace{0.5em}
        \begin{theorembox}{Rule of Thumb}
            Start lean, specialize as you scale. Full stack for speed, specialists for depth.
        \end{theorembox}
    \end{columns}
\end{frame}

% --- Slide 5: Cost Considerations ---
\begin{frame}{Engineering Compensation: Cost Factors}
    \framesubtitle{What drives salaries and total cost?}
    
    \vspace{0.3em}
    \begin{columns}[T]
        \column{0.48\textwidth}
        \textbf{\color{TuebingenRot}Role Premiums:}
        \begin{itemize}
            \setlength\itemsep{0.3em}
            \item AI/ML: Highest due to scarcity
            \item DevOps/SRE: Critical for scale
            \item Full stack: Versatility, less depth
        \end{itemize}
        
        \vspace{0.6em}
        \textbf{\color{TuebingenGold}Location Impact:}
        \begin{itemize}
            \setlength\itemsep{0.3em}
            \item Silicon Valley, NYC: +40--60\% cash
            \item Remote: Varies by company policy
            \item Europe: Generally lower base, better benefits
        \end{itemize}
        
        \column{0.48\textwidth}
        \textbf{\color{TuebingenGreen}Other Factors:}
        \begin{itemize}
            \setlength\itemsep{0.3em}
            \item Scarcity of specific skills
            \item Project complexity and scale
            \item Company stage (startup vs. enterprise)
            \item Equity/stock compensation
        \end{itemize}
        
        \vspace{0.6em}
        \begin{theorembox}{Hidden Costs}
            \footnotesize
            Mismatched skills $\rightarrow$ delays, rework, and blown budgets.\\
            Right talent = faster delivery, lower total cost.
        \end{theorembox}
    \end{columns}
\end{frame}


