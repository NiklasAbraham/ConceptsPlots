% Act I — Executive Grounding + AI History Timeline
\section{Executive Grounding and AI History}

% ==============================================================================
% 1.1 What do we mean by "AI" today? (2 slides)
% ==============================================================================

\begin{frame}{What Do We Mean by "AI" Today?}
    \framesubtitle{AI as an umbrella term}
    
    {\color{TuebingenGray} AI is not one thing—it's a spectrum of capabilities built on different techniques.}
    
    \vspace{0.5em}
    \begin{columns}[T]
        \column{0.48\textwidth}
        \textbf{The AI Umbrella Includes:}
        \begin{itemize}
            \setlength\itemsep{0.4em}
            \item \textbf{Rule-based automation} — explicit logic
            \item \textbf{Classical ML} — statistical learning from data
            \item \textbf{Deep learning} — neural networks at scale
            \item \textbf{Generative models} — content creation
            \item \textbf{Agentic systems} — autonomous action
        \end{itemize}
        
        \column{0.48\textwidth}
        \textbf{Executive Mental Model:}
        \begin{itemize}
            \setlength\itemsep{0.4em}
            \item Each layer builds on the previous
            \item More capability = more complexity
            \item Not all problems need the newest approach
            \item Match technique to problem
        \end{itemize}
    \end{columns}
    
    \vspace{1em}
    \begin{center}
        \small\color{TuebingenRot} \textbf{Key insight:} "AI" in your organization likely means multiple techniques coexisting.
    \end{center}
\end{frame}

\begin{frame}{The AI Capabilities Map}
    \framesubtitle{Three categories executives should remember}
    
    {\color{TuebingenGray} When evaluating AI initiatives, categorize by the type of capability being delivered.}
    
    \vspace{1em}
    \begin{columns}[T]
        \column{0.32\textwidth}
        \begin{center}
            {\Large \color{TuebingenRot} \textbf{Predictive}}
        \end{center}
        \vspace{0.3em}
        \begin{itemize}
            \setlength\itemsep{0.3em}
            \item Classification
            \item Forecasting
            \item Anomaly detection
            \item Risk scoring
        \end{itemize}
        \vspace{0.5em}
        {\small\color{TuebingenGray} "What will happen?"}
        
        \column{0.32\textwidth}
        \begin{center}
            {\Large \color{TuebingenGold} \textbf{Generative}}
        \end{center}
        \vspace{0.3em}
        \begin{itemize}
            \setlength\itemsep{0.3em}
            \item Text generation
            \item Code synthesis
            \item Image creation
            \item Document drafting
        \end{itemize}
        \vspace{0.5em}
        {\small\color{TuebingenGray} "Create something new"}
        
        \column{0.32\textwidth}
        \begin{center}
            {\Large \color{TuebingenGreen} \textbf{Agentic}}
        \end{center}
        \vspace{0.3em}
        \begin{itemize}
            \setlength\itemsep{0.3em}
            \item Tool use
            \item Multi-step reasoning
            \item Autonomous workflows
            \item Decision execution
        \end{itemize}
        \vspace{0.5em}
        {\small\color{TuebingenGray} "Act under constraints"}
    \end{columns}
    
    \vspace{1em}
    \begin{center}
        \small\color{TuebingenAnthrazit} \textbf{Governance complexity increases left to right →}
    \end{center}
\end{frame}

% ==============================================================================
% 1.2 What AI is not (1 slide)
% ==============================================================================

\begin{frame}{What AI Is \textit{Not}}
    \framesubtitle{Clearing misconceptions for better decisions}
    
    \vspace{0.5em}
    {\color{TuebingenRot}\textbf{AI is NOT:}}
    \begin{itemize}
        \setlength\itemsep{0.5em}
        \item \textbf{Human reasoning} — pattern matching, not understanding
        \item \textbf{Guaranteed truth} — probabilistic, can hallucinate
        \item \textbf{Deterministic} — same input can yield different outputs
        \item \textbf{A strategy substitute} — it's a capability, not a direction
        \item \textbf{Set-and-forget} — requires monitoring and maintenance
    \end{itemize}
\end{frame}

\begin{frame}{What AI \textit{Is}}
    \framesubtitle{A realistic view}
    
    \vspace{0.5em}
    {\color{TuebingenGreen}\textbf{AI IS:}}
    \begin{itemize}
        \setlength\itemsep{0.5em}
        \item \textbf{Statistical pattern recognition} at unprecedented scale
        \item \textbf{A tool} that amplifies human capability
        \item \textbf{Data-dependent} — quality in, quality out
        \item \textbf{An operational system} requiring governance
        \item \textbf{Rapidly evolving} — capabilities change quarterly
    \end{itemize}
    
    \vspace{0.5em}
    \begin{theorembox}{Executive Principle}
        Treat AI outputs as \textbf{drafts requiring verification}, not as \textbf{authoritative answers}. \\
        The value is in acceleration, not in abdication of judgment.
    \end{theorembox}
\end{frame}

% ==============================================================================
% 1.3 Why now (1 slide)
% ==============================================================================

\begin{frame}{Why Now? The Convergence}
    \framesubtitle{Four forces enabling the current AI moment}
    
    {\color{TuebingenGray} AI has had multiple hype cycles. What's different this time?}
    
    \vspace{0.3em}
    \begin{columns}[T]
        \column{0.24\textwidth}
        \begin{center}
            {\large \color{TuebingenRot} \textbf{1. Compute}}
        \end{center}
        \vspace{0.2em}
        \begin{itemize}
            \setlength\itemsep{0.2em}
            \item GPU acceleration
            \item Cloud scale
            \item 10,000× cheaper than 2012
        \end{itemize}
        
        \column{0.24\textwidth}
        \begin{center}
            {\large \color{TuebingenGold} \textbf{2. Data}}
        \end{center}
        \vspace{0.2em}
        \begin{itemize}
            \setlength\itemsep{0.2em}
            \item Internet-scale text
            \item Digitized operations
            \item Labeled datasets
        \end{itemize}
        
        \column{0.24\textwidth}
        \begin{center}
            {\large \color{TuebingenGreen} \textbf{3. Algorithms}}
        \end{center}
        \vspace{0.2em}
        \begin{itemize}
            \setlength\itemsep{0.2em}
            \item Transformers (2017)
            \item Transfer learning
            \item Scaling laws
        \end{itemize}
        
        \column{0.24\textwidth}
        \begin{center}
            {\large \color{TuebingenCyan} \textbf{4. Distribution}}
        \end{center}
        \vspace{0.2em}
        \begin{itemize}
            \setlength\itemsep{0.2em}
            \item API access
            \item IDE integration
            \item Consumer adoption
        \end{itemize}
    \end{columns}
    
    \vspace{0.6em}
    \begin{center}
        \begin{tikzpicture}
            \draw[->, thick, TuebingenRot] (0,0) -- (10,0);
            \foreach \x/\label in {0/2012, 2.5/2017, 5/2020, 7.5/2023, 10/2026} {
                \draw[TuebingenAnthrazit] (\x,0.1) -- (\x,-0.1) node[below] {\tiny\label};
            }
            \node[above, TuebingenAnthrazit] at (1.25,0.2) {\tiny AlexNet};
            \node[above, TuebingenAnthrazit] at (3.75,0.2) {\tiny Transformer};
            \node[above, TuebingenAnthrazit] at (6.25,0.2) {\tiny GPT-3};
            \node[above, TuebingenAnthrazit] at (8.75,0.2) {\tiny Enterprise AI};
        \end{tikzpicture}
    \end{center}
\end{frame}

% ==============================================================================
% 1.4 Full AI history timeline (12 slides with continuous timeline)
% ==============================================================================

% Define the timeline macro for consistent visualization across all slides
% Parameters: #1 = current era number (1-8)
\newcommand{\aitimeline}[1]{%
    \begin{tikzpicture}[remember picture]
        % Main timeline
        \draw[thick, TuebingenGray!50] (0,0) -- (11,0);
        
        % Era markers and labels
        \foreach \x/\era/\year/\label in {
            0.5/1/1940/A,
            2.0/2/1960/B,
            3.5/3/1970/C,
            5.0/4/1990/D,
            6.5/5/2006/E,
            8.0/6/2017/F,
            9.5/7/2020/G,
            11.0/8/2023/H%
        } {
            \ifnum\era=#1
                \fill[TuebingenRot] (\x,0) circle (5pt);
                \node[above=6pt, font=\tiny\bfseries, TuebingenRot] at (\x,0) {\year};
            \else
                \fill[TuebingenGray!40] (\x,0) circle (3pt);
                \node[above=4pt, font=\tiny, TuebingenGray] at (\x,0) {\year};
            \fi
        }
        
        % Era labels below
        \node[below=3pt, font=\tiny, TuebingenGray] at (0.5,0) {Found.};
        \node[below=3pt, font=\tiny, TuebingenGray] at (2.0,0) {Symb.};
        \node[below=3pt, font=\tiny, TuebingenGray] at (3.5,0) {Winter};
        \node[below=3pt, font=\tiny, TuebingenGray] at (5.0,0) {Stat.};
        \node[below=3pt, font=\tiny, TuebingenGray] at (6.5,0) {Deep};
        \node[below=3pt, font=\tiny, TuebingenGray] at (8.0,0) {Trans.};
        \node[below=3pt, font=\tiny, TuebingenGray] at (9.5,0) {Found.};
        \node[below=3pt, font=\tiny, TuebingenGray] at (11.0,0) {Now};
    \end{tikzpicture}%
}

% --- ERA A: Foundations (1940s–1960s) ---

\begin{frame}{The History of AI}
    \framesubtitle{Era A — Foundations (1940s–1960s)}
    
    \vspace{-0.3em}
    \begin{center}
        \aitimeline{1}
    \end{center}
    
    \vspace{0.5em}
    {\large\color{TuebingenRot}\textbf{The Birth of Artificial Intelligence}}
    
    \vspace{0.8em}
    \begin{columns}[T]
        \column{0.55\textwidth}
        \textbf{Key Milestones:}
        \begin{itemize}
            \setlength\itemsep{0.6em}
            \item \textbf{1943} — McCulloch \& Pitts: first neuron model \cite{mcculloch1943logical}
            \item \textbf{1950} — Turing's "Computing Machinery and Intelligence" \cite{turing1950computing}
            \item \textbf{1956} — Dartmouth: "AI" coined \cite{mccarthy1956dartmouth}
            \item \textbf{1958} — Rosenblatt's Perceptron \cite{rosenblatt1957perceptron}
        \end{itemize}
        
        \column{0.42\textwidth}
        \textbf{The Mood:}
        \begin{itemize}
            \setlength\itemsep{0.4em}
            \item Unbounded optimism
            \item "20 years to thinking machines"
            \item Heavy government funding
        \end{itemize}
        
        \vspace{0.8em}
        {\small\color{TuebingenGold}\textit{Lesson: Initial timelines were wildly optimistic.}}
    \end{columns}
\end{frame}

% --- ERA B: Symbolic AI (1960s–1970s) ---

\begin{frame}{The History of AI}
    \framesubtitle{Era B — Symbolic AI (1960s–1970s)}
    
    \vspace{-0.3em}
    \begin{center}
        \aitimeline{2}
    \end{center}
    
    \vspace{0.5em}
    {\large\color{TuebingenRot}\textbf{Logic-Based Reasoning}}
    
    \vspace{0.8em}
    \begin{columns}[T]
        \column{0.48\textwidth}
        \textbf{The Approach:}
        \begin{itemize}
            \setlength\itemsep{0.4em}
            \item Hand-coded expert rules
            \item Logic-based reasoning
            \item Knowledge representation
        \end{itemize}
        
        \column{0.48\textwidth}
        \textbf{1969 — Minsky \& Papert} \cite{minsky1969perceptrons}:
        \begin{itemize}
            \setlength\itemsep{0.4em}
            \item Critique of Perceptrons
            \item Showed fundamental limits
            \item Neural network funding collapsed
        \end{itemize}
    \end{columns}
    
    \vspace{0.8em}
    \begin{theorembox}{Why It Hit Limits}
        \textbf{Brittleness} — couldn't handle edge cases \quad | \quad \textbf{No learning} — couldn't improve from data
    \end{theorembox}
\end{frame}

% --- ERA C: AI Winters (1970s–1990s) - Split into 2 slides ---

\begin{frame}{The History of AI}
    \framesubtitle{Era C — First AI Winter (1974–1980)}
    
    \vspace{-0.3em}
    \begin{center}
        \aitimeline{3}
    \end{center}
    
    \vspace{0.5em}
    {\large\color{TuebingenRot}\textbf{Boom and Bust — Part I}}
    
    \vspace{0.8em}
    \begin{columns}[T]
        \column{0.48\textwidth}
        \textbf{The Collapse:}
        \begin{itemize}
            \setlength\itemsep{0.5em}
            \item DARPA cut funding after failed promises
            \item "AI can't deliver" sentiment spreads
            \item Research continued quietly in labs
        \end{itemize}
        
        \column{0.48\textwidth}
        \textbf{Expert Systems Boom (1980s):}
        \begin{itemize}
            \setlength\itemsep{0.5em}
            \item Commercial success initially
            \item XCON saved DEC \$40M/year
            \item Massive corporate investment
        \end{itemize}
    \end{columns}
    
    \vspace{1em}
    \begin{center}
        \color{TuebingenGold}\textbf{Pattern: Hype → Investment → Unmet expectations → Collapse}
    \end{center}
\end{frame}

\begin{frame}{The History of AI}
    \framesubtitle{Era C — Second AI Winter (1987–1993)}
    
    \vspace{-0.3em}
    \begin{center}
        \aitimeline{3}
    \end{center}
    
    \vspace{0.5em}
    {\large\color{TuebingenRot}\textbf{Boom and Bust — Part II}}
    
    \vspace{0.8em}
    \begin{columns}[T]
        \column{0.48\textwidth}
        \textbf{Why Expert Systems Failed:}
        \begin{itemize}
            \setlength\itemsep{0.5em}
            \item Expensive to maintain
            \item Rules became outdated quickly
            \item Couldn't adapt to change
            \item \$1B+ in failed projects
        \end{itemize}
        
        \column{0.48\textwidth}
        \textbf{Survivor Insight:}
        \begin{itemize}
            \setlength\itemsep{0.5em}
            \item The ideas weren't wrong
            \item Compute wasn't ready
            \item Data wasn't available
            \item Algorithms weren't mature
        \end{itemize}
    \end{columns}
    
    \vspace{1em}
    \begin{center}
        \color{TuebingenGreen}\textit{Sound familiar? The pattern would repeat.}
    \end{center}
\end{frame}

% --- ERA D: Statistical ML (1990s–2000s) ---

\begin{frame}{The History of AI}
    \framesubtitle{Era D — Statistical ML (1990s–2000s)}
    
    \vspace{-0.3em}
    \begin{center}
        \aitimeline{4}
    \end{center}
    
    \vspace{0.5em}
    {\large\color{TuebingenRot}\textbf{Learning from Data}}
    
    \vspace{0.8em}
    \begin{columns}[T]
        \column{0.48\textwidth}
        \textbf{Key Methods:}
        \begin{itemize}
            \setlength\itemsep{0.4em}
            \item Support Vector Machines \cite{cortes1995svm}
            \item Random Forests \cite{breiman2001randomforests}
            \item Boosting \cite{freund1997adaboost}
            \item Bayesian methods
        \end{itemize}
        
        \column{0.48\textwidth}
        \textbf{The Paradigm Shift:}
        \begin{itemize}
            \setlength\itemsep{0.4em}
            \item Don't encode rules — \textbf{learn patterns}
            \item More data → better models
            \item Practical: spam, fraud, recommendations
        \end{itemize}
    \end{columns}
    
    \vspace{0.8em}
    \begin{theorembox}{Executive Note}
        These techniques remain the right choice for many tabular/structured data problems today.
    \end{theorembox}
\end{frame}

% --- ERA E: Deep Learning (2006–2015) - Split into 2 slides ---

\begin{frame}{The History of AI}
    \framesubtitle{Era E — Deep Learning Revival (2006–2012)}
    
    \vspace{-0.3em}
    \begin{center}
        \aitimeline{5}
    \end{center}
    
    \vspace{0.5em}
    {\large\color{TuebingenRot}\textbf{Neural Networks Return}}
    
    \vspace{0.8em}
    \begin{columns}[T]
        \column{0.48\textwidth}
        \textbf{Key Breakthroughs:}
        \begin{itemize}
            \setlength\itemsep{0.5em}
            \item \textbf{2006} — Hinton's Deep Belief Networks \cite{hinton2006dbn}
            \item \textbf{2012} — \textcolor{TuebingenRot}{AlexNet} crushes ImageNet \cite{krizhevsky2012alexnet}
        \end{itemize}
        
        \column{0.48\textwidth}
        \textbf{What Changed:}
        \begin{itemize}
            \setlength\itemsep{0.5em}
            \item \textbf{GPUs} — 50× faster training
            \item \textbf{Big data} — ImageNet (14M images)
            \item \textbf{Open source} — reproducibility
        \end{itemize}
    \end{columns}
    
    \vspace{0.8em}
    \begin{theorembox}{The AlexNet Moment (2012)}
        Error rate dropped from 26\% to 15\% — a discontinuous leap proving deep learning works at scale.
    \end{theorembox}
\end{frame}

\begin{frame}{The History of AI}
    \framesubtitle{Era E — Deep Learning Expansion (2014–2015)}
    
    \vspace{-0.3em}
    \begin{center}
        \aitimeline{5}
    \end{center}
    
    \vspace{0.5em}
    {\large\color{TuebingenRot}\textbf{Going Deeper}}
    
    \vspace{0.8em}
    \begin{columns}[T]
        \column{0.48\textwidth}
        \textbf{Architecture Advances:}
        \begin{itemize}
            \setlength\itemsep{0.5em}
            \item \textbf{2014} — GANs \cite{goodfellow2014gan}
            \item \textbf{2015} — ResNet (152 layers!) \cite{he2016resnet}
            \item Dropout, batch norm, ReLU
        \end{itemize}
        
        \column{0.48\textwidth}
        \textbf{Key Lesson:}
        \begin{itemize}
            \setlength\itemsep{0.5em}
            \item When architecture + data + compute align...
            \item Progress can be sudden and dramatic
            \item Scale matters
        \end{itemize}
    \end{columns}
    
    \vspace{1em}
    \begin{center}
        \color{TuebingenGold}\textbf{The stage was set for language models.}
    \end{center}
\end{frame}

% --- ERA F: Transformers (2017–2020) - Split into 2 slides ---

\begin{frame}{The History of AI}
    \framesubtitle{Era F — The Transformer (2017)}
    
    \vspace{-0.3em}
    \begin{center}
        \aitimeline{6}
    \end{center}
    
    \vspace{0.5em}
    {\large\color{TuebingenRot}\textbf{"Attention Is All You Need"}}
    
    \vspace{0.8em}
    \begin{columns}[T]
        \column{0.48\textwidth}
        \textbf{The 2017 Paper} \cite{vaswani2017attention}:
        \begin{itemize}
            \setlength\itemsep{0.5em}
            \item Replaced sequential with parallel attention
            \item Enabled much larger models
            \item Better long-range dependencies
        \end{itemize}
        
        \column{0.48\textwidth}
        \textbf{What Followed:}
        \begin{itemize}
            \setlength\itemsep{0.5em}
            \item \textbf{2018} — BERT \cite{devlin2019bert}
            \item \textbf{2019} — GPT-2 \cite{radford2019gpt2}
            \item \textbf{2020} — GPT-3 \cite{brown2020gpt3}
        \end{itemize}
    \end{columns}
    
    \vspace{1em}
    \begin{center}
        \color{TuebingenGreen}\textbf{This is the architecture powering today's LLMs.}
    \end{center}
\end{frame}

\begin{frame}{The History of AI}
    \framesubtitle{Era F — Transfer Learning Paradigm (2018–2020)}
    
    \vspace{-0.3em}
    \begin{center}
        \aitimeline{6}
    \end{center}
    
    \vspace{0.5em}
    {\large\color{TuebingenRot}\textbf{Pretrain Once, Fine-tune Everywhere}}
    
    \vspace{0.8em}
    \begin{columns}[T]
        \column{0.48\textwidth}
        \textbf{The New Paradigm:}
        \begin{itemize}
            \setlength\itemsep{0.5em}
            \item Pretrain on massive text corpora
            \item Fine-tune for specific tasks
            \item Don't start from scratch
        \end{itemize}
        
        \column{0.48\textwidth}
        \textbf{Executive Implications:}
        \begin{itemize}
            \setlength\itemsep{0.5em}
            \item Language tasks suddenly tractable
            \item Foundation models as starting point
            \item Cost scales with model size
        \end{itemize}
    \end{columns}
    
    \vspace{0.8em}
    \begin{theorembox}{Key Insight}
        Context windows define capability limits. Larger context = more useful applications.
    \end{theorembox}
\end{frame}

% --- ERA G: Foundation Models (2020–2023) - Split into 2 slides ---

\begin{frame}{The History of AI}
    \framesubtitle{Era G — Foundation Models (2020–2022)}
    
    \vspace{-0.3em}
    \begin{center}
        \aitimeline{7}
    \end{center}
    
    \vspace{0.5em}
    {\large\color{TuebingenRot}\textbf{From Research to Products}}
    
    \vspace{0.8em}
    \begin{columns}[T]
        \column{0.48\textwidth}
        \textbf{Technical Advances:}
        \begin{itemize}
            \setlength\itemsep{0.5em}
            \item Instruction tuning \cite{ouyang2022instructgpt}
            \item RLHF — alignment with preferences
            \item Tool use — APIs, search, calculate
            \item Multimodal — text + images + code
        \end{itemize}
        
        \column{0.48\textwidth}
        \textbf{Key Releases:}
        \begin{itemize}
            \setlength\itemsep{0.5em}
            \item \textbf{ChatGPT} — 100M users in 2 months
            \item \textbf{GPT-4} \cite{openai2023gpt4} — multimodal
            \item \textbf{GitHub Copilot} — AI in IDEs
        \end{itemize}
    \end{columns}
    
    \vspace{1em}
    \begin{center}
        \color{TuebingenGold}\textbf{AI entered the mainstream consciousness.}
    \end{center}
\end{frame}

\begin{frame}{The History of AI}
    \framesubtitle{Era G — Enterprise Reality (2022–2023)}
    
    \vspace{-0.3em}
    \begin{center}
        \aitimeline{7}
    \end{center}
    
    \vspace{0.5em}
    {\large\color{TuebingenRot}\textbf{The Gap Between Demo and Production}}
    
    \vspace{0.8em}
    \begin{columns}[T]
        \column{0.48\textwidth}
        \textbf{Enterprise Challenges:}
        \begin{itemize}
            \setlength\itemsep{0.5em}
            \item Governance — who controls AI?
            \item Data privacy — where does data go?
            \item Integration — connecting to systems
            \item ROI — beyond demos to value
        \end{itemize}
        
        \column{0.48\textwidth}
        \textbf{Gaps Emerged:}
        \begin{itemize}
            \setlength\itemsep{0.5em}
            \item Demo-to-production is hard
            \item Hallucination is real
            \item Evaluation is immature
        \end{itemize}
    \end{columns}
    
    \vspace{0.8em}
    \begin{theorembox}{Executive Reality}
        The technology works. The challenge is making it work \textbf{reliably} in your context.
    \end{theorembox}
\end{frame}

% --- ERA H: Efficiency & Systems (2023–2026) - Split into 2 slides ---

\begin{frame}{The History of AI}
    \framesubtitle{Era H — Efficiency Revolution (2023–2024)}
    
    \vspace{-0.3em}
    \begin{center}
        \aitimeline{8}
    \end{center}
    
    \vspace{0.5em}
    {\large\color{TuebingenRot}\textbf{Better Models, Lower Costs}}
    
    \vspace{0.8em}
    \begin{columns}[T]
        \column{0.48\textwidth}
        \textbf{Efficiency Breakthroughs:}
        \begin{itemize}
            \setlength\itemsep{0.5em}
            \item Mixture-of-Experts \cite{jiang2024mixtral, deepseek2024moe}
            \item Distillation — small learns from large
            \item Quantization — less precision, same quality
        \end{itemize}
        
        \column{0.48\textwidth}
        \textbf{Open Models:}
        \begin{itemize}
            \setlength\itemsep{0.5em}
            \item Llama \cite{touvron2023llama}
            \item Mistral
            \item DeepSeek
            \item Self-hosting becomes viable
        \end{itemize}
    \end{columns}
    
    \vspace{1em}
    \begin{center}
        \color{TuebingenGreen}\textbf{Token costs dropped 100× in 2 years.}
    \end{center}
\end{frame}

\begin{frame}{The History of AI}
    \framesubtitle{Era H — Systems Discipline (2024–2026)}
    
    \vspace{-0.3em}
    \begin{center}
        \aitimeline{8}
    \end{center}
    
    \vspace{0.5em}
    {\large\color{TuebingenRot}\textbf{Where We Are Now}}
    
    \vspace{0.8em}
    \begin{columns}[T]
        \column{0.48\textwidth}
        \textbf{Systems Engineering:}
        \begin{itemize}
            \setlength\itemsep{0.5em}
            \item RAG \cite{lewis2020rag} — retrieval-augmented generation
            \item Evaluation discipline — measure first
            \item Workflow integration
            \item Agentic patterns with guardrails
        \end{itemize}
        
        \column{0.48\textwidth}
        \textbf{Executive Takeaway:}
        \begin{itemize}
            \setlength\itemsep{0.5em}
            \item Model selection = cost/capability trade-off
            \item Architecture > model size
            \item Evaluation is competitive advantage
        \end{itemize}
    \end{columns}
\end{frame}

\begin{frame}{Today's Reality}
    \framesubtitle{AI is a systems discipline, not just a model discipline.}
    
    \vspace{0.8em}
    \begin{theorembox}{Today's Reality}
        \textbf{AI is a systems discipline, not just a model discipline.}
    \end{theorembox}
\end{frame}

% ==============================================================================
% 1.5 Bridge: AI is a systems discipline now (1 slide)
% ==============================================================================

\begin{frame}{Bridge: AI Is a Systems Discipline Now}
    \framesubtitle{Setting up the deep dive}
    
    {\color{TuebingenGray} A working AI product is much more than a model.}
    
    \vspace{0.5em}
    \begin{center}
        \begin{tikzpicture}[
            box/.style={rectangle, draw=TuebingenAnthrazit, fill=TuebingenBeige, minimum width=2cm, minimum height=0.8cm, align=center, font=\small},
            arrow/.style={->, thick, TuebingenRot}
        ]
            % Boxes
            \node[box, fill=TuebingenRot!20] (data) at (0,0) {Data};
            \node[box, fill=TuebingenGold!20] (retrieval) at (2.5,0) {Retrieval};
            \node[box, fill=TuebingenGreen!20] (model) at (5,0) {Model};
            \node[box, fill=TuebingenCyan!20] (tools) at (7.5,0) {Tools};
            \node[box, fill=TuebingenGray!20] (eval) at (10,0) {Evaluation};
            
            % Arrows
            \draw[arrow] (data) -- (retrieval);
            \draw[arrow] (retrieval) -- (model);
            \draw[arrow] (model) -- (tools);
            \draw[arrow] (tools) -- (eval);
            
            % Feedback loop
            \draw[arrow, TuebingenGreen, dashed] (eval.south) -- ++(0,-0.5) -| (data.south);
            
            % Label
            \node[below, TuebingenGreen] at (5,-1) {\small Continuous improvement loop};
        \end{tikzpicture}
    \end{center}
    
    \vspace{0.3em}
    \begin{columns}[T]
        \column{0.48\textwidth}
        \textbf{The Full Stack:}
        \begin{itemize}
            \setlength\itemsep{0.2em}
            \item Data pipelines \& quality
            \item Retrieval \& knowledge systems
            \item Model selection \& prompting
            \item Tool integration \& guardrails
            \item Evaluation \& monitoring
        \end{itemize}
        
        \column{0.48\textwidth}
        \textbf{What This Means for You:}
        \begin{itemize}
            \setlength\itemsep{0.2em}
            \item Don't just "buy a model"
            \item Invest in data \& evaluation
            \item Architect for iteration
            \item Govern the whole system
        \end{itemize}
    \end{columns}
\end{frame}
